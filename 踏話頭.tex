\chapter{踏話頭}
臺灣是一个多元民族、多元語言的國家,
講母語、使用母語嘛是人上基本的權利,
毋過母語的電腦相關應用煞誠少。
臺灣本土語言百百種,
本論文先針對閩南語,
研究伊翻譯語料的特性,
除了閩南語本身以外,
嘛希望研究結果對別的本土語言有幫助。

本論文主要有五个貢獻。
第一个是本論文提出「拄好長度斷詞」的演算法,
雖然有改善「長詞優先」的缺點,
毋過效果無明顯,正確率差無$1\%$。
第二个是提出一个自動整理漢語語料的方法,
予資訊無完整的語料庫補足資訊,
發揮上大的價值,
BLEU分數對9.30鈕到13.82。
第三个貢獻是證明平行語料數量無到十萬句的時,
加語料對翻譯的效果影響非常大,
原本64121句加35025句了後,
BLEU分數對13.82加到19.33。
第四个貢獻是漢語語料用詞做單位去翻譯的時,
用斷字翻譯會當解決未知詞問題,
BLEU分數會當對18.64鈕到19.33。
上尾一个貢獻是提出分類兩種漢語的方法,
分類的錯誤率上低到$3.80\%$。

關鍵字:臺灣閩南語、華語、翻譯、語料、斷詞、語言分類
