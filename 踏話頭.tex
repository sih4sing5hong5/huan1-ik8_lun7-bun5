\chapter{踏話頭}
臺灣是一个多元民族、多元語言的國家。
講母語、使用母語是上基本的權利,
毋過母語的電腦相關應用煞誠少,
需要加強自然語言處理的研究佮語料收集整理。
臺灣本土語言百百種,
本論文是針對閩南語,
研究伊翻譯語料的特性。
除了閩南語本身以外,
嘛希望研究結果對別的本土語言有幫助。

本論文提出一个自動整理漢語語料的方法,
予資訊無完整的語料庫補足資訊,
發揮上大的價值,
BLEU分數對9.30搝到13.82。
另外閣用實驗證明平行語料數量無到十萬句的時,
加語料對翻譯的效果影響非常大,
原本64121句加到99147句了後,
BLEU分數對13.82提昇到19.33。

關鍵字:臺灣閩南語、華語、翻譯、語料、斷詞、語言分類
