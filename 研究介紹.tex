
\chapter{研究介紹}
\label{章:研究介紹}

咱的目標是予閩南語的翻譯,效果閣較好,
效果的好\ji{⿰禾黑}是看BLEU拍的分數\footnote{請看\ref{小節:評分方式}節的紹介}。

佇遮希望預處理語料,
予翻譯效果變好。
統計式機器翻譯的效果決定佇統計的模型,
若愛翻譯翻較好,有兩个大方向通做:
第一个方向是予語料的形式相像,
若語料的形式愈仝款,
翻譯的統計機率會閣較好。
第二个方向是資料愈濟愈好,
加新的語料庫了後,
翻譯模型有法度揀著上好的語詞來翻譯。

本章第\ref{節:閩南語斷詞}摻\ref{節:未知詞問題}節針對第一个方向做,
因為閩南語目前閣無剖析的程式,
本論文對齊模型用的是斷詞翻譯\footnote{請看\ref{節:翻譯}節的紹介},
語料的樣式就是以斷詞為主。
\ref{節:閩南語斷詞}節討論一種斷詞方法,
對閩南語數量無濟的語料上好的問題。
\ref{節:未知詞問題}節說明斷詞的語料翻譯,
一寡詞會翻袂出來的問題。
第二个方向是第\ref{節:整理語料}和\ref{節:分類語言}節,
\ref{節:整理語料}節講摻別的語料庫時,
會發生啥物問題。
\ref{節:分類語言}節對網路頂掠落來的資料,
愛按怎共語料照語言分類。


\section{閩南語斷詞}
\label{節:閩南語斷詞}
咱若共\ref{節:改變語料格式}節佮\ref{節:未知詞另外翻譯}節的方法合做伙,效果上好的是斷詞組對斷詞組,毋過斷詞組需要用剖析器去揣結構樹,閣來定規則決定詞佮詞啥物時陣愛敆做伙變詞組,這就是另外一門學問矣。目前閩南語閣無這資源,自然賰「華語斷詞-閩南語斷字」佮「華語斷詞-閩南語斷詞」上好,準若有好的斷詞工具,閩南語斷詞模型應該愛比閩南語斷字模型閣較好,雖然佇這擺實驗斷詞模型顛倒較\ji{⿰禾黑},毋過為著未來閩南語的斷詞研究方便比較,後壁的實驗模型攏是用「華語斷詞-閩南語斷詞」。
%------------------
加入新聞語料庫、教育部辭典佮數位典藏了後,按呢華臺平行語料有98814句\footnote{新聞語料庫64121句,教育部辭典34693句},會當訓練語言模型的閩南語有\footnote{新聞語料庫64121句,教育部辭典例句34693句、附錄句388句,數位典藏416343句},這个數量對照別種語言語料庫的數量也是小可嫌少。



\section{未知詞問題}
\label{節:未知詞問題}
系統結構會當看圖,語言模型用Witten-Bell加discounting的算法,翻譯模型用預設的參數。

訓練語料用新聞語料庫頭前2300篇新聞,攏總57167句,試驗語料用上尾267篇新聞,攏總6954句。按呢無調整語料,直接照伊的斷詞組落去訓練,共結果佮答案一句內底拆做一字一字,用的BLEU去算分數,得著70.67分。

詳細看分數歹的原因,是因為傷濟詞組佇訓練語料無出現過,親像提原本試驗語料的華語句「陸續 開放 一百五十項 的 規費」去翻譯,得著「liok8-siok8 khai1-hong3 一百五十項 e5 規費」 ,「一百五十項」無翻譯出來,是因為訓練語料內底無出現過這个詞組,對訓練語料來講,「一百五十項」就是一个未知詞組。但是訓練語料內底有「兩項」佮「一百五十位」的華語詞組,煞無法度提來用。

為著予翻譯的結果閣較好,按算用兩種方式來加強未知詞的處理,頭一个是改變翻譯的單位,共原本斷詞組的語料改做斷詞抑是斷字,寫佇\ref{節:改變語料格式}節。第二个方法仝款照斷詞組翻譯,若拄著未知詞,針對未知詞專工處理,會當看\ref{節:未知詞另外翻譯}節。

\section{整理語料}
\label{節:整理語料}
\section{分類語言}
\label{節:分類語言}