\chapter{研究介紹}
\label{章:研究介紹}

咱的目標是予閩南語的翻譯,效果閣較好,
效果的好\ji{⿰禾黑}是看BLEU拍的分數\footnote{請看\ref{小節:評分方式}節的紹介}。

佇遮希望預處理語料,
予翻譯效果變好。
統計式機器翻譯的效果決定佇統計的模型,
若愛翻譯翻較好,有兩个大方向通做:
第一个方向是予語料的形式相像,
若語料的形式愈仝款,
翻譯的統計機率會閣較好。
第二个方向是資料愈濟愈好,
加新的語料庫了後,
翻譯模型有法度揀著上好的語詞來翻譯。

本章第\ref{節:閩南語斷詞}摻\ref{節:未知詞問題}節針對第一个方向做,
因為閩南語目前閣無剖析的程式,
本論文對齊模型用的是斷詞翻譯\footnote{請看\ref{節:翻譯}節的紹介},
語料的樣式就是以斷詞為主。
\ref{節:閩南語斷詞}節討論佗一種斷詞方法,
對數量無濟的閩南語語料上好。
\ref{節:未知詞問題}節說明斷詞的語料翻譯,
一寡詞會翻袂出來的問題。
第二个方向是第\ref{節:整理語料}和\ref{節:語言分類}節,
\ref{節:整理語料}節講摻樣式無仝的語料庫時,
會發生啥物問題。
\ref{節:語言分類}節對網路頂掠落來的資料,
愛按怎共語料照語言分類。

\section{閩南語斷詞}
\label{節:閩南語斷詞}
愛用斷詞的形式來翻譯,
華語有中研院的中文斷詞系統,
閩南語無現成的系統。
所以頭一个問題就是閩南語欲按怎斷詞,
而且比較無仝的斷詞方法,
\ji{⿰因}的效果分別是按怎。

\section{未知詞問題}
\label{節:未知詞問題}
用斷詞做語料單位,
親像有的華語詞無出現佇語料過,
翻譯模型毋知伊愛對應到佗一个閩南語詞,
就會翻袂出來。
因為訓練語料無可能有全部的華語,
就愛想一个辦法,處理有詞翻袂出來的情形。
%加例
\section{整理語料}
\label{節:整理語料}
完整的閩南語語料應該有全漢、全羅佮斷詞三種資訊,
毋過誠少有語料庫資訊完整,
三種資訊攏有,
而且逐个語料庫資訊狀況攏無仝。
第三个問題就是討論按怎利用手頭有的語料庫,
去整理無完整的語料庫,
予翻譯的效果閣較好。

%共漢羅全羅一字一字對齊了後,會發覺一个問題,有的字是一對一,有的字煞干焦音標爾。

%加例
\section{語言分類}
\label{節:語言分類}
增加語料庫的一个方法就是去網路頂掠閩南語的資料,
毋過網頁頂的閩南語定定佮華語濫做伙,
所以愛揣一个方法分類這兩个語言。
先前大部份語言分類的研究攏是拼音文字,
用字元的語言模型去判斷,
毋過閩南語佮華語誠濟用詞是仝款的,
用字的語言模型去判斷效果無好,
所以第四个問題就是閣欲加揣啥物款的特徵,
會當來幫助語言分類。
