\chapter{研究介紹}
\label{章:研究介紹}

阮的目標是予閩南語的翻譯,效果閣較好,
效果的好\ji{⿰禾黑}是看BLEU拍的分數\footnote{請看\ref{小節:翻譯評分方式}節的紹介}。

佇遮希望預處理語料,
予翻譯效果變好。
統計式機器翻譯的效果決定佇統計的模型,
若愛翻譯翻較好,有兩个大方向通做:
第一个方向是予語料的形式相像,
若語料的形式愈仝款,
翻譯的統計機率會閣較好。
第二个方向是資料愈濟愈好,
加新的語料庫了後,
翻譯模型揀著好的語詞來翻譯的機會愈大。

本章第\ref{節:閩南語斷詞}摻\ref{節:未知詞問題}節針對第一个方向做,
因為閩南語目前閣無剖析的程式,
本論文對齊模型用的是斷詞翻譯\footnote{請看\ref{節:翻譯}節的紹介},
語料的樣式就是以斷詞為主。
\ref{節:閩南語斷詞}節討論佗一種斷詞方法,
對數量無濟的閩南語語料上好。
\ref{節:未知詞問題}節說明斷詞的語料翻譯,
一寡詞會翻袂出來的問題。
第二个方向是第\ref{節:整理語料}和\ref{節:語言分類}節,
\ref{節:整理語料}節講摻樣式無仝的語料庫時,
會發生啥物問題。
\ref{節:語言分類}節對網路頂掠落來的資料,
愛按怎共語料照語言分類。

\section{閩南語斷詞}
\label{節:閩南語斷詞}
愛用斷詞的形式來翻譯,
華語有中研院的中文斷詞系統,
閩南語無現成的系統。
所以頭一个問題就是閩南語欲按怎斷詞,
而且比較無仝的斷詞方法,
\ji{⿰因}的效果分別是按怎。

\section{未知詞問題}
\label{節:未知詞問題}
用斷詞做語料單位,
親像有的華語詞無出現佇語料過,
翻譯模型毋知伊愛對應到佗一个閩南語詞,
就會翻袂出來。
因為訓練語料無可能有全部的華語,
親像表\ref{表:未知詞問題範例}的例就無「憂愁」華語詞的翻譯方法,
就愛想一个辦法,處理有詞翻袂出來的情形。


\begin{table}
\caption{未知詞問題範例}
\label{表:未知詞問題範例}
\centering
\begin{tabular}{ccc}
\hline
\multirow{2}{*}{訓練語料1} & 自稱 一輩子 離不開 預報 工作 的 吳德榮 , \\
 & tsu7-tshing1 tsit8-si3-lang5 li5-be7-khui1… \\
\hline
\multirow{2}{*}{訓練語料2} & 開始 傷愁 了 , \\
 & khai1-si2 siong1-tshiu5 ah4 , \\
\hline
\multicolumn{2}{c}{…} \\
\hline
\hline
試驗輸入 & 一輩子 吃 穿 都 不用 憂愁 了 \\
翻譯結果 & tsit8-si3-lang5 tsiah8 tshing7 long2 m7-bian2 憂愁 \\
\hline
\end{tabular}
\end{table}

%訓練語料1
%自稱 一輩子 離不開 預報 工作 的 吳德榮 ,
%tsu7-tshing1 tsit8-si3-lang5 li5-be7-khui1…
%\tsoo{自}{⿳⿳ㄗㄨ˫}{tsū}
%\tsoo{稱}{⿳⿳ㄑㄧㄥ}{tshing}
%\tsoo{一}{⿳⿳⿳ㄐㄧ㆐ㆵ}{tsi̍t}
%\tsoo{世}{⿳⿳ㄒㄧ˪}{sì}
%\tsoo{人}{⿳⿳ㄌㄤˊ}{lâng}
%\tsoo{離}{⿳⿳ㄌㄧˊ}{lî}
%\tsoo{袂}{⿳⿳ㆠㆤ˫}{bē}
%\tsoo{開}{⿳⿳ㄎㄨㄧ}{khui} …
%訓練語料2
%開始 傷愁 了 ,
%khai1-si2 siong1-tshiu5 ah4
%\tsoo{開}{⿳ㄎㄞ}{khai}
%\tsoo{始}{⿳⿳ㄒㄧˋ}{sí}
%\tsoo{傷}{⿳⿳ㄒㄧㆲ}{siong}
%\tsoo{愁}{⿳⿳⿳ㄑㄧㄨˊ}{tshiû}
%\tsoo{矣}{⿳ㄚㆷ}{ah}
%,
%......
%試驗輸入 一輩子 吃 穿 都 不用 憂愁 了
%翻譯結果 tsit8-si3-lang5 tsiah8 tshing7 long2 m7-bian2 憂愁

\section{整理語料}
\label{節:整理語料}
完整的閩南語語料應該有全漢、全羅佮斷詞三種資訊,
毋過誠少有語料庫三種資訊攏有,
而且逐个語料庫資訊狀況攏無仝。

第三个問題就是討論按怎利用手頭有的語料庫,
補好全漢、全羅佮斷詞三種資訊,
狀況親像表\ref{表:整理語料語料庫狀況},
希望整理了的語料庫,
會當予翻譯的效果閣較好。

%轉格式--
%辭典
%彼 个 查某 囡仔 真媠 。
%Hit e5 tsa-boo2 gin2-a2 tsin sui2 .
%嘛向望 老母 身體 勇起來
%ma7-ng3-bang7 lau7-bu2 sin1-the2 iong2-khi2-lai5
%臺文典藏
%Koh m7知u7危險
%Koh m7-tsai u7 gui5-hiam2

\begin{table}
\caption{語料庫狀況}
\label{表:整理語料語料庫狀況}
\centering
\begin{tabular}{cccc}
 & 全漢 & 全羅 & 斷詞 \\
教育部辭典 & O & O & O \\
新聞語料庫 & O & O & X\tablefootnote{因為新聞語料斷詞無規範} \\
臺文典藏 & X & O\tablefootnote{臺文典藏少數無全羅} & O \\
\end{tabular}
\end{table}

\section{語言分類}
\label{節:語言分類}

\begin{table}
\caption{語言分類範例}
\label{表:語言分類範例}
\centering
\begin{tabular}{p{30em}l}
語句 & 語言\\
\hline
聽人講 khah 早有出現過『小蜜蜂』 & 閩南語\\
我 beh tńg 來種作 ! ── 記 0312 Truku 反亞泥 ‧ 還我土地運動 & 閩南語\\
有台灣味 ê 繪本──《我和我的腳踏車》 . & 閩南語\\
「 糟了 ,是工地火燒厝, 緊轉去打 火 ! 」建設公司 的 工地主任從手機接到消息,通話結束後就帶著那群混混先離開了。 & 華語\\
去越南胡志明市 4 工/越南胡志明市四日行 @Gio̍k-hōng & 華語\\
\end{tabular}
\end{table}

增加語料庫的一个方法就是去網路頂掠閩南語的資料,
毋過網頁頂的閩南語定定佮華語濫做伙,
所以愛揣一个方法分類這兩个語言,
可比講表\ref{表:語言分類範例}按呢。
先前大部份語言分類的研究攏是拼音文字,
用字元的語言模型去判斷,
毋過閩南語佮華語誠濟用詞是仝款的,
用字的語言模型去判斷效果無好,
所以第四个問題就是閣欲加揣啥物款的特徵,
會當來幫助語言分類。


