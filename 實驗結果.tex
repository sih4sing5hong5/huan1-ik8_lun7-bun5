\chapter{實驗結果}
\label{章:實驗結果}

\begin{table}
\caption{實驗工具版本}
\label{表:實驗工具版本}
\centering
\begin{tabular}{l|l}
工具 & 版本\\
\hline
臺灣言語工具\cite{臺灣言語工具} & 0.5.0\\
Moses\cite{Koehn:2007:MOS:1557769.1557821} & commit 40c819d285cdeb40c0b8cc428bfde2fcb531b655\\
GIZA++\cite{och2003systematic} & 1.0.7\\
SRILM\cite{stolcke2002srilm} & 1.7.0\\
\end{tabular}
\end{table}

為著逐家後壁研究的方便,
本論文研究的程式佮結果,
全部公開佇網路頂的專案\cite{翻譯研究}。
開發工具的版本會當看表\ref{表:實驗工具版本},
詳細的參數,
可比講語言模型用Witten-Bell加discounting的算法,
翻譯模型用預設的訓練包攏會當佇內底的設定看著。
\section{閩南語斷詞實驗}
\label{節:閩南語斷詞實驗}

\begin{table}
\caption{閩南語斷詞的效果}
\label{表:閩南語斷詞的效果}
\centering
\begin{tabular}{c|ccc}
斷詞方法 & 召回率 & 精確率 & F測量\\
\hline
拄好長度斷詞 & 91.1 & 85.1 & 88.0\\
長詞優先斷詞(對頭前) & 91.0 & 84.9 & 87.9\\
長詞優先斷詞(對後壁) & 91.1 & 85.0 & 88.0\\
\end{tabular}
\end{table}

本實驗是提教育部辭典的35130个詞條當做訓練語料,
試驗語料是教育部辭典例句8027句。
共試驗語料的斷詞資訊提掉了後,
用拄好長度斷詞佮長詞優先去斷詞,
才閣佮原本例句的斷詞比較,
得著表\ref{表:閩南語斷詞的效果}的結果。

會當看著拄好長度斷詞的分數有比長詞優先閣較好淡薄,
毋過無明顯的進步。
這兩種斷詞方法的精確率攏比召回率低誠濟,
代表辭典內底收的詞閣無夠濟。

\section{語料整理實驗}
\label{節:語料整理實驗}

\begin{table}
\caption{新聞語料庫佮數位典藏互相整理的實驗}
\label{表:互相整理實驗}
\centering
\begin{tabular}{lcccccc}
整理幾擺 & 原始語料 & 1 & 2 & 3 & 4 & 5\\
BLEU分數 & 9.30 & 14.72 & 13.77 & 13.82 & 13.82 & 13.82\\
\end{tabular}
\end{table}

完整的閩南語語料愛有全漢、全羅佮斷詞資訊。
因為新聞語料庫有全漢、全羅,無斷詞,數位典藏有斷詞毋過無完整的全漢佮全羅。
%,教育部辭典斷詞佮全漢全羅攏有。
%攏就會使用教育部辭典佮數位典藏共新聞語料庫斷詞,閣共斷好的新聞語料庫佮教育部辭典提來標數位典藏的一對一,閣重做幾仔擺,到收斂為止,親像圖XX。

本實驗是提教育部辭典的35130个詞條佮附錄388句當做標準語料,
用拄好長度斷詞來整理新聞平行語料64121句佮數位典藏329476句。
而且用詞為單位拍分數。

表\ref{表:互相整理實驗}是整理的結果,
整理的結果一息仔就收斂,
會當看著整理了的分數比猶未整理前好欲一半。
毋過整理第二擺了後,分數有降一寡,
看整理了的結果,
是因為新聞佮典藏內底的攏有一寡錯誤,
所以第二擺用著遮的資料,
會影響著整理的結果。

%人-為|jin5-ui5 的|e5
%人-為-的|jin5-ui5-e5


\section{分類語言實驗}
\label{節:判斷語言實驗}

\begin{figure}
\caption{無仝特徵詞數量,分類3741段閩南語華語}
\label{圖:無仝特徵詞數量對分類閩南語華語效果的影響}
\begin{tikzpicture}
\begin{axis}[
scaled y ticks=real:1,
ytick scale label code/.code={},
ymax = 14,
symbolic x coords={0,10,20,50,100,200,500,1000,2000,3000},
xtick=data,
height=8cm,
width=14cm,
grid=major,
xlabel={特徵詞數量},
ylabel={分類錯誤率},
legend style={
cells={anchor=east},
legend pos=north east,
%mark size=0.5em
}
]
\addplot coordinates {
(0,13.79) (10,6.87) (20,5.45) (50,4.12)
(100,3.88) (200,3.90) (500,4.12)
(1000,3.80) (2000,4.14) (3000,4.14)
%(0,516) (10,257) (20,204) (50,154)
%(100,145) (200,146) (500,154)
%(1000,142) (2000,155) (3000,155)
};

\legend{SVM}
\end{axis}
\end{tikzpicture}
\end{figure}

這節實驗的語料是對TGB通訊創刊開始,到2014年6月12日為止攏總177期1179篇文章,
提出頭前1000篇做訓練語料,閩南語有9368段488844詞,華語有8519段439436詞;
後壁179篇做訓練語料,閩南語有1344段75282詞,華語有2397段114901詞。
以段做辨識單位,提來予支援向量機分類。

毋過6012个特徵實在是傷濟矣,所以咱試看覓共3000特徵詞減少,看會影響著辨識率無。
實驗結果佇表\ref{圖:無仝特徵詞數量對分類閩南語華語效果的影響},
佇50~100个特徵詞分類效果就收斂矣,加閣較濟的特徵詞,無啥影響著辨識的效果。

\section{加入TGB語料庫實驗}
\label{節:加入TGB語料庫實驗}

\begin{table}
\caption{加入TGB語料的翻譯效果}
\label{表:加入TGB語料的翻譯效果}
\centering
\begin{tabular}{lcccccc}
& 加TGB語料前 & 加TGB語料後\\
平行語料句數 & 64121句 & 99146句\\
BLEU分數 & 13.82 & 19.33\\
\end{tabular}
\end{table}

頂一節做分類語言的實驗,
紲落來就是共TGB語料摻入來翻譯語料。

先提教育部詞條佮附錄句和
\ref{節:語料整理實驗}節整理了的新聞佮典藏語料
來整理TGB語料,
閣用Bleualign來對齊,
就會使提著35025句TGB平行語料。

實驗訓練語料除了\ref{節:語料整理實驗}節的語料外,
閣加入TGB平行語料,
除了教育部附錄句、典藏干焦會使做語言模型以外,
原本平行語料干焦是新聞語料庫64121句,
加入TGB語料35025句了後變做99146句,
對表\ref{表:加入TGB語料的翻譯效果}會當看著進步誠濟,
代表平行語料閣無夠,
閣佇加語料就會使幫助翻譯的階段。
%母語愛專注佇語料處理

\section{斷詞樣式佮斷字樣式的翻譯結果實驗}
\label{節:斷詞樣式佮斷字樣式的翻譯結果實驗}


\begin{figure}
\caption{斷字佮斷詞語料的翻譯效果比較}
\label{圖:斷字佮斷詞語料的翻譯效果比較}
\begin{tikzpicture}
    \begin{axis}[
        width  = 0.85*\textwidth,
        height = 8cm,
        major x tick style = transparent,
        x tick label style={rotate=30,anchor=east},
        ybar=2*\pgflinewidth,
        bar width=14pt,
        ymajorgrids = true,
        ylabel = {BLEU分數},
        symbolic x coords={華語斷字-閩南語斷字,華語斷字-閩南語斷詞,華語斷詞-閩南語斷字,華語斷詞-閩南語斷詞},
        xtick = data,
        enlarge x limits=0.25,
%        scaled y ticks = false,
       % ymin=0,
%        legend cell align=left,
%        legend style={
%                at={(1,1.05)},
%                anchor=south east,
%                column sep=1ex
%        }
    ]
        \addplot
            coordinates {(華語斷字-閩南語斷字, 31.85) (華語斷字-閩南語斷詞,31.24)
  	          (華語斷詞-閩南語斷字,30.74) (華語斷詞-閩南語斷詞,29.22)};
        \addplot
            coordinates {(華語斷字-閩南語斷字, 31.85) (華語斷字-閩南語斷詞,31.26)
  	          (華語斷詞-閩南語斷字,31.90) (華語斷詞-閩南語斷詞,30.92)};
        \addplot
            coordinates {(華語斷字-閩南語斷字, 31.85) (華語斷字-閩南語斷詞,31.24)
  	          (華語斷詞-閩南語斷字,31.44) (華語斷詞-閩南語斷詞,30.43)};
        \legend{無處理未知詞,斷字-斷字翻譯,斷字-斷詞翻譯}
    \end{axis}
\end{tikzpicture}
\end{figure}

為著愛比較華語佮閩南語

這个實驗的訓練、試驗語料佮頂一个實驗是仝款的,
只是上尾是用字做單位來算BLEU分數,
所以圖\ref{圖:斷字佮斷詞語料的翻譯效果比較}的實驗分數看起來會較懸,
並毋是翻譯的效果較好。

這个實驗會當看著兩件代誌,
第一件是未知詞若有處理,
對翻譯有幫助,
而且未知詞用「華語斷字-閩南語斷字」的效果比「華語斷字-閩南語斷詞」閣較好,
可能是因為未知詞大部份攏是需要一字一字照翻的,
嘛有可能是語料無夠濟,
斷字對斷字的統計數量較濟。

第二件代誌是華語的斷詞對翻譯有幫助,
毋過閩南語的斷詞煞無。
可能是閩南語的斷詞閣無夠準,
造成翻譯效果變\ji{⿰禾黑}。

這个實驗是用字做比較單位,
雖然華語斷詞配閩南語斷字的成績上好,
若佇需要斷詞的環境下\footnote{親像語音合成},
閣愛算落斷詞效果可能無好,
效果會拍折。