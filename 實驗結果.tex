
\chapter{實驗結果}
\label{章:實驗結果}


\section{整理實驗流程佮結果}
\label{節:整理實驗流程佮結果}

因為新聞語料庫有一對一需要斷詞,數位典藏有斷詞毋過愛轉一對一\footnote{因為有一部份是干焦漢字,所以嘛愛標音標},教育部辭典斷詞佮一對一攏有。
攏就會使用教育部辭典佮數位典藏共新聞語料庫斷詞,閣共斷好的新聞語料庫佮教育部辭典提來標數位典藏的一對一,閣重做幾仔擺,到收斂為止,親像圖XX。

實驗語料除了\ref{節:未知詞問題}節分配新聞語料庫,嘛加入教育部辭典的例句\footnote{原本8XXX句,替換腔口詞了變34693句,詳細請看附錄\ref{章:腔口統計佮處理}}。
所以訓練語料是用新聞語料庫頭前57167句佮教育部31200句,
試驗語料是新聞語料庫後壁6954句佮教育部3493句例句。
語言模型除了用訓練語料以外,閣有加教育部附錄句388句。

拍分數以詞為單位

\begin{table}
\caption{新聞語料庫佮數位典藏互相整理的實驗結果}
\label{表:互相整理實驗結果}
\centering
\begin{tabular}{lcccc}
整理幾擺 & 0\tablefootnote{新聞只用教育部辭典斷詞} & 1 & 2 & 3\\
語言模型無加典藏 & 50.57 & 53.91 & 53.91 & 53.91\\
語言模型加典藏 & 49.50 & 51.94 & 51.97 & 51.97\\
\end{tabular}
\end{table}
 佇使用仝款的語料時

典藏佮新聞有互相整理過,分數有較懸
 摻典藏做語言模型


分數降落來
因為典藏對訓練佮試驗語料來講是外部的資料
 分數一息仔就收斂



新聞語料做第一擺就收斂
典藏到第二擺就收
\section{判斷語言實驗結果}
\label{節:判斷語言實驗結果}
這節實驗的語料是對TGB通訊創刊開始,到2014年6月12日為止攏總177期1179篇文章,
提出頭前1000篇做訓練語料,閩南語有9368段488844詞,華語有8519段439436詞;
後壁179篇做訓練語料,閩南語有1344段75282詞,華語有2397段114901詞。
以段做辨識單位,提來做SVM,錯誤率是 $\%$ 。

毋過14012个特徵實在是傷濟矣,所以咱試看覓共7000特徵詞減少,看會影響著辨識率無。
而且PCA佮LDA是轉特徵空間到較細的空間,來試配合SVM效果會按怎,結果會當看圖\ref{圖:無仝分類模型佮特徵詞數量對分類臺華語效果的影響}。
對結果來看SVM比LDA效果閣較好,用PCA對SVM效果無啥物傷大的影響。
而且對50~100个特徵詞了後,加閣較濟的特徵詞,攏袂影響的辨識的效果。

\begin{figure}
\begin{tikzpicture}
\begin{axis}[
scaled y ticks=real:1,
ytick scale label code/.code={},
ymax = 14,
symbolic x coords={0,10,20,50,100,200,500,1000,2000,3000},
xtick=data,
height=12cm,
width=16cm,
grid=major,
xlabel={特徵詞數量},
ylabel={分類錯誤率},
legend style={
cells={anchor=east},
legend pos=north east,
%mark size=0.5em
}
]
\addplot coordinates {
(0,13.79) (10,6.87) (20,5.45) (50,4.12)
(100,3.88) (200,3.90) (500,4.12)
(1000,3.80) (2000,4.14) (3000,4.14)
%(0,516) (10,257) (20,204) (50,154)
%(100,145) (200,146) (500,154)
%(1000,142) (2000,155) (3000,155)
};

\legend{SVM}
\end{axis}
\end{tikzpicture}
\caption{無仝特徵詞數量,分類3741段臺華語的效果}
\label{圖:無仝分類模型佮特徵詞數量對分類臺華語效果的影響}
\end{figure}
