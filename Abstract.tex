\chapter{Abstract}
%Southern Min, Taiwanese, is major local language in Taiwan.
%臺灣是一个多元民族、多元語言的國家,
Taiwan is a multi-culture and multi-language country.
%講母語、使用母語嘛是人上基本的權利,
Speaking in mother tongues is a basic human right,
%毋過母語的相關應用煞誠少。
but there are few computer applications for mother languages.
%需要加強自然語言處理的研究佮語料收集整理。
The applications are supported by corpus and research of natural language processing.
%臺灣本土語言百百種,
There are many local languages in Taiwan.
%本論文先針對閩南語,
This thesis focuses on Southern Min Taiwanese, is major local language in Taiwan.
%研究伊翻譯語料的特性,
It contains research into corpus preprocessing to get good performance in statistical machine translation.
%除了閩南語本身以外,
%嘛希望研究結果對別的本土語言有幫助。
We wish it can help the computational linguistic research of other local language of Taiwan.

%本論文提出一个自動整理漢語語料的方法,
%予資訊無完整的語料庫補足資訊,
This thesis introduces a method to preprocess the corpus whose information is lacking.
%發揮上大的價值,
%BLEU分數對9.30鈕到13.82。
After refining,
the BLEU score is raised from 9.30 to 13.82.
%第三个貢獻是證明平行語料數量無到十萬句的時,
%加語料對翻譯的效果影響非常大,
Experiments in this thesis show that
translation performance is sensitive to the amount of parallel corpus
when the amount of parallel corpus sentences is less than 100,000.
%原本64121句加35025句了後,
%BLEU分數對13.82加到19.33。
The BLEU score raises from 13.82 to 19.33 as the amount of sentences increased from 64121 to 99147.

%關鍵字:臺灣閩南語、華語、翻譯、語料、斷詞、語言分類
Keywords: Southern Min, Taiwanese, Mandarin, Chinese, Translation, Corpus, Segmentation, Language Identification