\chapter{Abstract}
%Southern Min, Taiwanese, is major local language in Taiwan.
%臺灣是一个多元民族、多元語言的國家,
Taiwan is a multi-culture and multi-language country.
%講母語、使用母語嘛是人上基本的權利,
Saying mother tongues is a basic human right,
%毋過母語的相關應用煞誠少。
But there are a few electronic applications for these languages.
%臺灣本土語言百百種,
There are many local languages in Taiwan.
%本論文先針對閩南語,
This paper focuses on Southern Min, Taiwanese, is major local language in Taiwan.
%研究伊翻譯語料的特性,
It contains research into corpus preprocessing to get good performance in statistical machine translation,
%除了閩南語本身以外,
%嘛希望研究結果對別的本土語言有幫助。
and it is also expected help for other Taiwan languages.%%

%本論文主要有五个貢獻。
There are five major contributions in this paper.
%第一个是本論文提出「拄好長度斷詞」的演算法,
First, this paper introduces the "Appropriate Length Segmentation",
%雖然有改善「長詞優先」的缺點,
which improves shortage of "Maximum Matching".
%毋過效果無明顯,正確率差無$1\%$。
It is better than Maximum Matching,
but performance between these segmentation methods is below $1\%$.
%第二个是提出一个自動整理漢語語料的方法,
Second, we refine the corpus,
%予資訊無完整的語料庫補足資訊,
whose information is lacking.
%發揮上大的價值,
%BLEU分數對9.30鈕到13.82。
After refining,
the BLEU score is raised from 9.30 to 13.82.
%第三个貢獻是證明平行語料數量無到十萬句的時,
%加語料對翻譯的效果影響非常大,
Third, the experiment says
translation performance is sensitive to the amount of parallel corpus
when the amount of parallel corpus sentences is less than 100,000.
%原本64121句加35025句了後,
%BLEU分數對13.82加到19.33。
The BLEU scores are 13.82 and 19.33 when the amount of sentences are 64121 and 99147 respectively.
%第四个貢獻是漢語語料用詞做單位去翻譯的時,
%用斷字翻譯會當解決未知詞問題,
Forth, translating the Chinese word out of vocabulary by Chinese character.
%BLEU分數會當對18.64鈕到19.33。
The BLEU score is raised from 18.64 to 19.33.
%上尾一个貢獻是提出分類兩種漢語的方法,
Finally, we introduce a identification method for two Chinese languages.
%分類的錯誤率上低到$3.80\%$。
It achieves $3.80\%$ error rate.

%關鍵字:臺灣閩南語、華語、翻譯、語料、斷詞、語言分類
Keyword:Southern Min, Taiwanese, Mandarin, Chinese, Translation, Corpus, Segmentation, Language Identification