\chapter{結論佮未來發展}
\label{章:結論佮未來發展}
目前臺灣的本土語言佇沓沓仔流失,
需要逐家用心來注意,
這嘛是這篇論文研究華語翻譯到閩南語的上主要目的。
下跤會盡量共經驗分享予其他母語,
希望閩南語以外,
客話佮咱的原住民南島語,
嘛會使利用這个研究成果。

\section{結論}
\label{節:結論}
本論文提出「拄好長度斷詞」的方法,
試圖改善「長詞優先」的缺點,
毋過佇\ref{節:閩南語斷詞實驗}節的實驗內底,
對無濟訓練語料來講,
兩種斷詞方法的效果略略仔爾。

%斷詞的影響
%客話

佇\ref{節:語料整理實驗}節佮\ref{節:加入TGB語料庫實驗}節的實驗,
會當看著整理閩南語的效果,
對原本的9.30分加到19.33分,
代表對閩南語來講,
這馬無夠十萬句的語料,
是翻譯效果\ji{⿰禾黑}的一大限制。
對臺灣別的本土語言來講,
語料數量是一个上大的問題。

分類華語佮閩南語兩種語言佇\ref{節:判斷語言實驗}節嘛有$96\%$正確的成績,
若愛一擺分類三種以上n个漢語,
會使簡化做兩種漢語的分類,
先共n个語言,
兩个兩个語言彼此之間揣出特徵詞,
訓練出$\frac{n\times(n-1)}{2}$个分類模型,
上尾投票,
看佗一个語言贏較濟,
就當作是彼个語言。

可比講這馬有閩南語、客家四縣話、客家饒平話、佮華語四種語言愛分類,
先訓練「閩南語-客家四縣話」、「閩南語-客家饒平話」、「閩南語-華語」、
「客家四縣話-客家饒平話」、「客家四縣話-華語」、「客家饒平話-華語」六个模型,
閣來共試驗語料提入來予六个模型判斷,
分別判斷出「客家四縣話」、「客家饒平話」、「閩南語」、
「客家饒平話」、「客家四縣話」、「客家饒平話」,
看這六个模型分類內底「客家饒平話」上濟,
就判斷做「客家饒平話」的語料。
準做有需要分類南島語,會當看\ref{小節:語言分類}節的說明。

紲落來講兩个會當加強翻譯的方法,佮一个翻譯模型會當閣利用的所在:
\section{機器校對}
\label{節:鬥相共人工校對}

%圖:鬥相共人工校對
\ref{節:語料整理實驗}節的實驗結果共咱講,
就算咱提著的語料毋是蓋完整,
咱嘛會當補足伊的資訊,
予翻譯變好。
這嘛表示語料正確度對翻譯有影響,
用機器整理語料有伊的限制,
若愛予語料正確,
一定就需要人工校對。

%抑是用有一對一較完整的資料去鬥處理,攏會使予翻譯的效果閣較好。
%毋過完整的資料數量若較濟,翻譯的效果會愈好。

%準做資料是親像數位典藏仝款補一部份漢字爾仔,嘛是愛人工閣巡一擺,其他漢羅抑是全羅的用電腦自動標一對一了後,閣較需要人工共⿰因閣校對過。
%愛有完整一對一資料,人工是走袂去的,親像圖X仝款,咱有完整的資料,去標猶未整理的語料,經過人工檢查了後,完整的資料就閣較濟,按呢標一對一就閣較準,就無需要遮爾濟的人工,創造一个好的循環。

毋過人工校對是開錢開時間開氣力的代誌,
準做有機器校對系統,
予資料的正確率閣較懸,
就會減少人工檢查的負擔,
嘛會當閃避無彩工,一直改仝款的語料錯誤。

所以做一个即時更新(Online)的機器校對系統就是重要的問題,
這个問題準做有$n$組人工檢查前的錯誤語料$t_{i}$佮人工檢查了的標準語料$p_{i}$,
決定一開始的訓練語料數量$m$組,
希望紲落來第$m+1,m+2,...,n$組攏會當用進前校對過的資料來做機器校對,
予機器校對的結果$p'_{i}$改做$p_{i}$的人工校對功夫上少,
會當定義寫做公式\ref{公式:機器校對幫助人工校對公式}。

\begin{equation}
\label{公式:機器校對幫助人工校對公式}
\begin{split}
\sum\limits_{i=m+1}^n \{ p'_{i}編輯到p_{i}的人工校對成本\},\\
其中p'_{i}是t_{i}機器校對的結果,\\
機器校對是(t_{1},p_{1}),(t_{2},p_{2}),...,(t_{i-1},p_{i-1})訓練的
\end{split}
\end{equation}

%人工檢查前的輸入佮人工檢查了的輸出,
%嘛會使做一个改錯字的系統,
%予人免一直改仝款的錯誤。
%機器整理模型就愛綴咧改,
%予錯誤較少,
%毋過嘛有可能予原本整理著的語料變做錯誤,
%所以會當閣準備一个鬥相共的工具,
%予整理模型改變時,
%原本整理著的語料袂變毋著,
%才有法度。

%毋管是標全漢全羅,
%抑是斷詞、剖析佮語音資料,
發展技術佮照顧語料對發展臺灣母語的研究來講平平仔重要,
絕對袂使重視技術煞袂記得語料。

%\subsection{重斷數位典藏}
%\label{節:重斷數位典藏}
%數位典藏有的斷詞其實毋好,親像「」應該是兩个詞

\section{斷詞}
\label{節:未來斷詞}

華語有夠濟的斷詞標記語料,
所以華語斷詞是一个發展足完整的技術。
毋過閩南語的語料無華語遐爾濟的標記,
就會影響著斷詞效果。

本論文是用對「長度優先」的算法改做「拄好長度斷詞」,
除了這个方法以外,嘛會使閣用統計方法看斷詞的效果會較好無。
用統計的方法會使用翻譯工具來做,
一字一字斷字的輸入配合一詞一詞斷詞的輸出,
提去訓練翻譯模型,按呢就有一个統計的斷詞工具。

到底佗一个方法對閩南語、客話這種只有幾萬、幾十萬句語料,
效果會閣較好,
就需要另外研究矣。


\section{字幕辨識}
\label{節:字幕辨識}
臺灣母語的語音資料其實誠濟,
親像電視劇、廣播攏有誠濟的語音資訊,
只毋過遮的聲音語料大部份攏是配華語字幕。
若有一个工具會當補母語字幕,
咱就會當用這母語資訊來教學,
抑是會來做語言學佮自然語言處理的研究。

\begin{table}
\caption{字幕辨識問題分析}
\label{表:字幕辨識問題分析}
\centering
\begin{tabular}{c|c|c}
 & 輸入 & 輸出 \\
\hline
語音辨識 & 臺灣母語的語音 & 臺灣母語的字幕 \\
\hline
翻譯 & 華語的字幕 & 臺灣母語的字幕 \\
\hline
字幕辨識 & \begin{tabular}[x]{@{}c@{}}臺灣母語的語音\\華語的字幕\end{tabular} & 臺灣母語的字幕 \\
%http://tex.stackexchange.com/questions/2441/how-to-add-a-forced-line-break-inside-a-table-cell
\end{tabular}
\end{table}

字幕辨識的問題就是輸入「臺灣母語的語音」佮「華語的字幕」,
想欲輸出「臺灣母語的字幕」。
會當看表\ref{表:字幕辨識問題分析},
其實這个問題是綜合語音辨識佮翻譯的問題,
語音辨識會當對「臺灣母語的語音」提供字詞佮先後的資訊,
翻譯會當對「華語的字幕」提供詞的機率,
語言模型會當提供語句的合理性,
就需要另外的研究看這三个物件按怎用合理的數學方式鬥起來。