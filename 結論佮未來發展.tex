\chapter{結論佮未來發展}
\label{章:結論佮未來發展}
目前臺灣的本土語言佇沓沓仔流失,
需要逐家用心來注意,
這嘛是這篇論文研究華語翻譯到閩南語的上主要目的。
下跤會盡量共經驗分享予其他母語,
希望閩南語以外,
客話佮咱的原住民南島語,
嘛會使利用這个研究成果。

\section{結論}
\label{節:結論}
本論文提出「拄好長度斷詞」的方法,
試圖改善「長詞優先」的缺點,
毋過佇\ref{節:閩南語斷詞實驗}節的實驗內底,
對無濟訓練語料來講,
兩種斷詞方法的效果略略仔爾。

%斷詞的影響
%客話

佇\ref{節:語料整理實驗}節佮\ref{節:加入TGB語料庫實驗}節的實驗,
會當看著整理閩南語的效果,
對原本的9.30分加到19.33分,
代表對閩南語來講,
這馬無夠十萬句的語料,
是翻譯效果\ji{⿰禾黑}的一大限制。
對臺灣別的本土語言來講,
語料數量是一个上大的問題。

分類華語佮閩南語兩種語言佇\ref{節:判斷語言實驗}節嘛有$96\%$正確的成績,
若愛一擺分類三種以上n个漢語,
會使簡化做兩種漢語的分類,
先共n个語言,
兩个兩个語言彼此之間揣出特徵詞,
訓練出$\frac{n\times(n-1)}{2}$个分類模型,
上尾投票,
看佗一个語言贏較濟,
就當作是彼个語言。

可比講這馬有閩南語、客家四縣話、客家饒平話、佮華語四種語言愛分類,
先訓練「閩南語-客家四縣話」、「閩南語-客家饒平話」、「閩南語-華語」、
「客家四縣話-客家饒平話」、「客家四縣話-華語」、「客家饒平話-華語」六个模型,
閣來共試驗語料提入來予六个模型判斷,
分別判斷出「客家四縣話」、「客家饒平話」、「閩南語」、
「客家饒平話」、「客家四縣話」、「客家饒平話」,
看這六个模型分類內底「客家饒平話」上濟,
就判斷做「客家饒平話」的語料。
準做有需要分類南島語,會當看\ref{小節:語言分類}節的說明。

紲落來講兩个會當加強翻譯的方法,佮一个翻譯模型會當閣利用的所在:
\section{鬥相共人工校對}
\label{節:鬥相共人工校對}

%圖:鬥相共人工校對
\ref{節:語料整理實驗}節的實驗結果共咱講,
就算咱提著的語料毋是蓋完整,
咱嘛會當補足伊的資訊,
予翻譯變好。
這嘛表示語料正確度對翻譯有影響,
用機器整理語料有伊的限制,
若愛予語料正確,
一定就需要人工校對。

%抑是用有一對一較完整的資料去鬥處理,攏會使予翻譯的效果閣較好。
%毋過完整的資料數量若較濟,翻譯的效果會愈好。

%準做資料是親像數位典藏仝款補一部份漢字爾仔,嘛是愛人工閣巡一擺,其他漢羅抑是全羅的用電腦自動標一對一了後,閣較需要人工共⿰因閣校對過。
%愛有完整一對一資料,人工是走袂去的,親像圖X仝款,咱有完整的資料,去標猶未整理的語料,經過人工檢查了後,完整的資料就閣較濟,按呢標一對一就閣較準,就無需要遮爾濟的人工,創造一个好的循環。

毋過人工校對是開錢開時間開氣力的代誌,
人工校對上驚無彩工一直改仝款的語料錯誤,
機器整理模型就愛綴咧改,
予錯誤較少,
毋過嘛有可能予原本整理著的語料變做錯誤,
所以會當閣準備一个鬥相共的工具,
予整理模型改變時,
原本整理著的語料袂變毋著,
才有法度。

%親像圖XX
準做有改錯字工具,
予標一對一的正確率閣較懸,
就會減少人工檢查的負擔。
人工檢查前的輸入佮人工檢查了的輸出,
嘛會使做一个改錯字的系統,
予人免一直改仝款的錯誤。

毋管是標全漢全羅,
抑是斷詞、剖析佮語音資料,
發展技術佮照顧語料對發展閩南語研究來講平平仔重要,
絕對袂使重視技術煞袂記得語料。

%\subsection{重斷數位典藏}
%\label{節:重斷數位典藏}
%數位典藏有的斷詞其實毋好,親像「」應該是兩个詞

\section{斷詞}
\label{節:未來斷詞}
%看楊允言
華語斷詞是一个發展足完整的技術\footnote{CKIP正確率%},毋過閩南語的語料毋華語遐爾濟的人工,就會影響著效果。
本論文是用對「上長度優先」的算法改做「拄好長度斷詞」\footnote{請看\ref{節:拄好長度斷詞}節},嘛會使閣用統計方法看斷詞的效果會較好無。
用統計的方法會使用翻譯工具來做,一字一字斷字的輸入配合一詞一詞斷詞的輸出,提去訓練翻譯模型,按呢就有一个統計的斷詞工具。
到底佗一个方法佇閩南語、客話這種幾萬、幾十萬句語料,效果會閣較好,就需要閣一寡研究矣。
\section{剖析}
\label{節:未來剖析}
%看楊允言
楊允言教授捌做過閩南語的剖析,毋過需要語言學的知識,閣有一致的人工檢查,而且一開始的語料歹收集,可能著的先共閩南語翻譯做華語,才閣共華語語句的剖析結果\footnote{中研院剖析}對應轉去閩南語語句,按呢就有初步的資料矣。
訂好規則了後,就需要訂好剖析的規則
人工若校對,全部的資料著愛收集起來
而且嘛愛親像XX節的XX圖仝款,愛做一个改錯誤的程式,予人工莫一直改仝款的物件
等待資料有夠濟,就會提樹仔的語料來訓練一个閩南語的剖析器\footnote{star ford parser}


\section{語音辨識}
\label{節:未來辨識}
字幕 華語字→閩南語字