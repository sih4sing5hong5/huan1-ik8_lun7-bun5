\chapter{結論佮未來發展}
\label{章:結論佮未來發展}

\section{結論}
\label{節:結論}
斷詞的影響

佇\ref{節:語料整理實驗}節佮\ref{節:加入TGB語料庫實驗}節的實驗,
會當看著整理閩南語的效果,
對原本的9.30分加到19.33分,
代表對閩南語來講,
這馬無夠十萬句的語料,
是翻譯效果\ji{⿰禾黑}的一大限制。



目前臺灣的本土語言佇沓沓仔流失,需要逐家用心來注意,這嘛是這篇論文研究華語翻譯到閩南語的上主要目的。
為著逐家後壁研究的方便,本論文研究的過程佮結果,全部公開佇網路頂\footnote{\url{https://github.com/sih4sing5hong5/huan1-ik8_gian5-kiu3.git}},詳細按怎用請看附錄\ref{章:按怎裝程式}。
除了閩南語以外,客話佮咱的原住民南島語,嘛會使利用這个研究成果。
若是閣配合語音模型\footnote{請看附錄\ref{章:語音模型}},都會使做一个即時口語翻譯系統。
若第一線佇學校教冊的老師需要電子化的翻譯,抑是關心母語的文字工作者,攏會使利用這个研究成果繼續做落去。

後壁的XX章節是針對閩南語所寫的研究方向。XX是客話、南島話攏會使參考的…
\section{校對資料}
\label{節:校對資料}
面頂的實驗結果共咱講就算咱提著的語料毋是蓋完整,咱嘛會當利用改伊的款,抑是用有一對一較完整的資料去鬥處理,攏會使予翻譯的效果閣較好。
毋過完整的資料數量若較濟,翻譯的效果會愈好。

準做資料是親像數位典藏仝款補一部份漢字爾仔,嘛是愛人工閣巡一擺,其他漢羅抑是全羅的用電腦自動標一對一了後,閣較需要人工共⿰因閣校對過。
愛有完整一對一資料,人工是走袂去的,親像圖X仝款,咱有完整的資料,去標猶未整理的語料,經過人工檢查了後,完整的資料就閣較濟,按呢標一對一就閣較準,就無需要遮爾濟的人工,創造一个好的循環。

毋過一開始人工檢查是開錢開時間開氣力的代誌,若準做有工具,予標一對一的正確率閣較懸,就會減少人工檢查的負擔。
人工檢查前的輸入佮人工檢查了的輸出,嘛會使做一个改錯字的系統,予人免一直改仝款的錯誤。

毋管是標一對一,抑是下跤的斷詞、剖析佮語音資料,發展技術佮照顧語料對發展閩南語研究來講平平仔重要,絕對袂使重視技術煞袂記得語料。
%\subsection{重斷數位典藏}
%\label{節:重斷數位典藏}
%數位典藏有的斷詞其實毋好,親像「」應該是兩个詞

\section{斷詞}
\label{節:未來斷詞}
%看楊允言
華語斷詞是一个發展足完整的技術\footnote{CKIP正確率%},毋過閩南語的語料毋華語遐爾濟的人工,就會影響著效果。
本論文是用對「上長度優先」的算法改做「拄好長度斷詞」\footnote{請看\ref{節:拄好長度斷詞}節},嘛會使閣用統計方法看斷詞的效果會較好無。
用統計的方法會使用翻譯工具來做,一字一字斷字的輸入配合一詞一詞斷詞的輸出,提去訓練翻譯模型,按呢就有一个統計的斷詞工具。
到底佗一个方法佇閩南語、客話這種幾萬、幾十萬句語料,效果會閣較好,就需要閣一寡研究矣。
\section{剖析}
\label{節:未來剖析}
%看楊允言
楊允言教授捌做過閩南語的剖析,毋過需要語言學的知識,閣有一致的人工檢查,而且一開始的語料歹收集,可能著的先共閩南語翻譯做華語,才閣共華語語句的剖析結果\footnote{中研院剖析}對應轉去閩南語語句,按呢就有初步的資料矣。
訂好規則了後,就需要訂好剖析的規則
人工若校對,全部的資料著愛收集起來
而且嘛愛親像XX節的XX圖仝款,愛做一个改錯誤的程式,予人工莫一直改仝款的物件
等待資料有夠濟,就會提樹仔的語料來訓練一个閩南語的剖析器\footnote{star ford parser}


\section{語音辨識}
\label{節:未來辨識}
字幕 華語字→閩南語字