\documentclass[final,oneside,onecolumn,12pt,a4paper]{book}%
%薛丞宏加的
\usepackage{fontspec} 
\usepackage{xeCJK} 
\XeTeXlinebreaklocale "zh" 
\XeTeXlinebreakskip = 0pt plus 1pt 
\pagestyle{empty}
%Select fonts
%\setmainfont[Mapping=tex-text]{Times New Roman} % rm
%\setsansfont[Mapping=tex-text]{Arial}           % sf
%\setmonofont{Courier New}                       % tt
\setCJKmainfont{DFKai-SB} %xelatex 標楷體
\setCJKmonofont{MingLiU}  %xelatex 細明體
\linespread{3}

\usepackage[toc,page]{appendix}%加附錄
\usepackage{makecell}%予表的第一格會當有一條舒舒的線隔做兩格
\usepackage{tablefootnote}%佇表內底用註解footnote in table
\usepackage{pgfplots}%畫圖表的套件

\usepackage{colortbl}
\usepackage{ruby}

\usepackage{graphicx}
%\usepackage[export]{adjustbox}

%\definecolor{intnull}{RGB}{213,229,255}

\renewcommand{\rubysep}{-4ex}

\makeatletter%予臺語音標會當佇字下跤
\newcommand{\rubybot}[2]{
  \@tempdimc \f@size\p@
  \begin{tabular}[t]{@{}>{\hspace{-6pt}}l<{\hspace{-9pt}}@{}}
%	\arrayrulecolor{red}\hline
%    #1\\[-1.7em]
	#1\\[-1.7em]
    \fontsize{.5\@tempdimc}{.5\@tempdimc}\selectfont%
    \setlength{\normalbaselineskip}{0pt}#2 
    %\cellcolor{intnull}
    
%	\\\arrayrulecolor{blue}\hline
  \end{tabular}%
}
\makeatother

%\fboxrule=4pt%border thickness

\usepackage{ifthen}

\newcommand{\too}[1]{\raisebox{-.2\height}{\includegraphics[height=1em]{#1}}}

\newcommand{\ji}[1]{\too{字/#1}}

\newcommand{\im}[1]{\too{音/#1}}
%\raisebox{-.2\height}{\includegraphics[height=1em]{音/#1}}

\newcommand{\tsoo}[3]{
\rubybot{#1\ifthenelse{\equal{#2}{}}{}{\im{#2}}}{#3}
}
%薛丞宏加的

\usepackage{amsmath}
\usepackage{amsfonts}
\usepackage{amssymb}
\usepackage{url}
\usepackage{hyperref}
\usepackage{algorithm}
\usepackage{algorithmic}
\usepackage{graphicx}%
\setcounter{MaxMatrixCols}{30}
\usepackage[left=3cm, right=2cm, top=2.5cm, bottom=2.5cm]{geometry}
%TCIDATA{OutputFilter=latex2.dll}
%TCIDATA{Version=5.50.0.2953}
%TCIDATA{Created=Monday, May 12, 2003 22:46:51}
%TCIDATA{LastRevised=Friday, August 30, 2013 14:39:59}
%TCIDATA{<META NAME="GraphicsSave" CONTENT="32">}
%TCIDATA{<META NAME="SaveForMode" CONTENT="1">}
%TCIDATA{BibliographyScheme=BibTeX}
%TCIDATA{<META NAME="DocumentShell" CONTENT="Standard LaTeX\Blank - Standard LaTeX Article">}
%TCIDATA{Language=American English}
%TCIDATA{PageSetup=72,72,72,72,0}
%TCIDATA{Counters=arabic,1}
%TCIDATA{AllPages=
%H=36
%F=36
%}
%BeginMSIPreambleData
\providecommand{\U}[1]{\protect\rule{.1in}{.1in}}
%EndMSIPreambleData
\oddsidemargin 0.0in
\textheight=8.5in
\textwidth=6.5in
\headheight=0.0in
\topmargin=0.0in
\newtheorem{theorem}{Theorem}
\newtheorem{abstract}{abstract}
\newtheorem{acknowledgement}[theorem]{Acknowledgement}
\newtheorem{axiom}[theorem]{Axiom}
\newtheorem{case}[theorem]{Case}
\newtheorem{claim}[theorem]{Claim}
\newtheorem{conclusion}[theorem]{Conclusion}
\newtheorem{condition}[theorem]{Condition}
\newtheorem{conjecture}[theorem]{Conjecture}
\newtheorem{corollary}[theorem]{Corollary}
\newtheorem{criterion}[theorem]{Criterion}
\newtheorem{definition}[theorem]{Definition}
\newtheorem{example}[theorem]{Example}
\newtheorem{exercise}[theorem]{Exercise}
\newtheorem{lemma}[theorem]{Lemma}
\newtheorem{notation}[theorem]{Notation}
\newtheorem{problem}[theorem]{Problem}
\newtheorem{proposition}[theorem]{Proposition}
\newtheorem{remark}[theorem]{Remark}
\newtheorem{solution}[theorem]{Solution}
\newtheorem{summary}[theorem]{Summary}
\interdisplaylinepenalty=2500
\sloppy
\pagenumbering{arabic}
\pagestyle{plain}
\renewcommand{\baselinestretch}{2}
\begin{document}
\begin{titlepage}

\begin{center}

   

\textsc{\Huge 國 立 交 通 大 學} %38
\\[2em]
\textsc{\LARGE 資訊科學與工程研究所} %28
\\[2em]
\textsc{\LARGE 碩 士 論 文} %28
\\[3em]

% Title
{\huge \bfseries 臺灣閩南語翻譯語料自動整理 } %20
\\[1em]
{\LARGE Auto-refining Southern Min, Taiwanese, Corpus for Translations}
\\[3em]

\begin{table}[H]
\centering
\Large
\begin{tabular}{ll}
研 究 生: & 薛丞宏\\ %18
指導教授: & 張智星教授\\ %18
 & 易志偉教授\\ %18
\end{tabular}
\end{table}

\vfill

{\large 中 華 民 國  103  年  9  月}

\end{center}

\end{titlepage}


\frontmatter
\chapter{踏話頭}
\tsoo{臺}{⿳⿳ㄉㄞˊ}{tai5}
\tsoo{語}{⿳⿳ㆣㄧˋ}{gi2}
媠\footnote{\tsoo{臺}{⿳⿳ㄉㄞˊ}{tai5}}
臺灣閩南語是臺灣本土語言上大的族群之一,
這馬是應用程式為主的時代,誠濟應用攏必須倚靠佇技術面頂,
所以這篇論文分三部份討論語料庫的處理對華語翻譯做臺灣閩南語的影響。

第一部份是語料型式對翻譯效果的影響,佇有處理未知詞的情形下,斷詞組的效果比斷詞佮斷字效果閣較好。

第二部份有整理猶未整理的語料,用兩三種語料提供的無仝資訊互相整理,看整理了對翻譯效果的影響。

第三部份是討論網路語料定定共臺灣閩南語佮北京腔華語兩種語言濫做伙,
針對這種情形按怎分別兩種語言,用啥物特徵佮用啥物辨識模型,
上尾特徵詞數量100个左右就會當得著袂禾黑的效果,支持向量機的效果上好。

關鍵字:臺灣閩南語、華語、翻譯、語料庫、斷詞、語言判斷
\newpage

\chapter{摘要}
臺灣閩南語是臺灣本土語言的族群之一

\newpage

\chapter{Abstract}

Thank you

\newpage

\chapter{勞力}

\newpage

\tableofcontents
\listoffigures
\listoftables

\mainmatter

\chapter{研究背景}
\label{章:研究背景}
臺灣是多元民族的國家,逐家攏有\ji{⿰因}家己的文化,逐个民族講的話攏無仝,有人講閩南語,有人講泰雅語,有人講客話,……,最近二十幾年來閣有越南話、印尼話、……新住民的語言。臺灣語言主要會當分做南島語佮漢語兩種\footnote{新住民的越南語是南亞語系,泰國話是狀侗語}。

南島語是全世界分布上闊的語族\footnote{tte wt臺灣、紐西蘭、},南島語會當分做四个大族群\cite{李壬癸}。咱臺灣攏總有這四種族群\footnote{加上印尼?},所以臺灣的原住民毋干焦一个語系。
\includegraphics[height=1em]{字/⿰因}閣比漢人較早到臺灣,才會予人叫原住民。
佇歷史的文書記錄面頂,十七世紀荷蘭佮西班牙來臺灣進前就有一个幾仔族原住族組合起來的大肚王國\cite{大肚王國},
\ji{⿰因}保護\ji{⿰因}家己的土地,抵抗荷蘭人,鄭成功政權,清國的統治,到大甲西社事件才滅國。
到今日,臺灣的原住民除了中華民國政府的原民會認定的十六族以外,閣有誠濟猶未認定的族,親像臺南的西拉雅佮埔里的噶哈巫……,攏總有二三十族以上。


臺灣用的漢語主要會當分做三大種,閩南語、客家話佮官話\cite{外省族群的母語與國語}。頭一種閩南語是對四百外年前開始,佇明國、清國的福建人因為枵腹肚蹛袂落,姑不而將駛船渡過烏水溝\footnote{又叫臺灣海峽},來臺灣趁食。佇清國的時陣透過政治佮經濟的力量一步一步食掉原住民的土地,上尾佇臺灣變做上大的族群之一。

定定有人問:「閩南語是欲按怎寫!?」除了用拼音的方式寫出來以外,閩南語八九成是有漢字的\cite{愛揣},%洪惟仁
鶴佬人\footnote{就是閩南人,這用字有爭議,教育部無訂落來,採用洪惟仁教授的建議\cite{愛揣}}祖先搬去閩越地區時,就佮遐的壯苗族原住民通婚,因為原住民人濟,所以佇予鶴佬人同化的過程就留落來這一兩成毋是漢語的詞\cite{董忠司},親像「查甫」佮「查某」的「查」\footnote{雖然大部份人攏讀「tsa」,毋過佇臺灣漢語辭典\cite{臺灣漢語辭典}有記錄「ta」的音,佮「大」仝音},「大家」佮「大官」的「大」是狀侗族的詞頭。

閩南語的漢語部份是西晉後尾動亂的兩擺大移民、唐朝陳元光𤆬兵鎮壓狀苗族原住族,佮南宋時期文教影響,攏總四擺移民,制度影響,造成四層漢語語言層的閩南語\cite{閩客方言史稿}。
頭前三層語言層的語音叫做白話音,上尾第四擺後南宋音號做文讀音,親像「石」這个字,有「石頭」、「石榴」\footnote{一種果子}、「藥石」\footnote{方藥與砭石兩種藥仔}三種語音,佇語言是規律變化的假設之下,這个「石」字就代表上少有三層語言層。

臺灣第二个漢語客家話嘛是佇明清時代的客人生活過袂落,坐船來臺灣。這馬佇臺灣較大腔口有「四海大平安」\footnote{四縣腔、海陸腔、大埔腔、饒平腔、詔安腔}。
鶴佬人佮客人毋管佇亞洲大陸抑是臺灣,生活攏無遠,誠濟辭的用法攏相像,親像閩南語講「頭前」,客話講「」\footnote{客話拼音會當看客話拼音\cite{客話拼音},意思會當查客話辭典\cite{客話辭典}},閩南語講「好勢」,客話嘛講「好勢」。

佇臺灣的鶴佬人佮客人來臺灣佮原住民通婚,生活中嘛濫著原住民的用詞,親像「臺灣」是西拉雅語「Taian」或「Tayan」對外地人的稱呼\cite{台灣名稱的由來},咱定定食著的「菝仔」嘛是平埔族話\cite{客語外來語}。

毋但原住民話,閩南語佮客話嘛有佮別的語言交插,像是的「六甲地」的「甲」是從荷蘭「akker」來的\cite{台甲}。
受過日本的統治,閩南語佮客話攏有濫著日語,日語「トマトtomato」親像「臭柿仔」閩南語閣會當講「kha7*-ma1*-tooh*」\footnote{音標有*代表外來詞,頭前兩音節kha7、ma1免閣變調},客話講「」,「オ-トバイootobai」閩南語講「機車」「oo1-too」,客話講。
會當講是臺灣的歷史攏藏佇臺灣的語言內底。

上尾一个漢語官話是中華民國政府拍輸中國人民共和國了,𤆬誠濟中國人來臺灣,因為逐个的故鄉攏無仝\cite{外省族群的母語與國語},民國政府就繼續用佇亞洲大陸的規範,共北京官話的語音、白話文的用法當做基礎,利用政府機關,教育認知\footnote{本人阿姨有佇學校講母語予人罰過錢}、…等手法佇臺灣捒,所以這馬是佇臺灣是有上政治優勢的語言。
毋過這馬佇臺灣實際用的官話閣佮北京用的官話無仝款,有加入臺灣本土的元素,是北京官話的次方言之一。
除了這三種漢語以外,佇二次大戰了中華民國政府𤆬來臺灣的人,\ji{⿰因}的母語嘛誠濟種,毋過這馬攏已經消失甲差不多矣。

為著方便起見,本論文下跤的閩南語佮客話攏是講佇臺灣用的閩南語佮客話,
官話就照中國海外華人的慣習,稱呼佇臺灣使用的北京官話次方言做華語。
中華民國政府就暫時叫做用政府,中華民國政府的教育部號做教育部。%愛閣順過
本論文主要用閩南語寫,毋過會用著一寡華語佮客話。
為著格式一致,予人有法度一看就知影是啥物語言,
閩南語會用「臺羅拼音\cite{}佮方言音符號\cite{}」,客話會用「客話拼音\cite{}」,華語會用「注音符號\cite{}」。
引用的閩南語的羅馬拼音攏會轉臺羅,毋過人名、文章名、冊名保持原樣。

\section{研究目的}
\label{節:研究目的}
%母語重要→資料無夠→需要翻譯→整理翻譯語料
最近十幾冬政府開始注重人權,講母語是人上基本的權利,
毋但國校仔國中攏開始有鄉土語言的課矣\footnote{除了本土語言以外,閣有新住民語言},
逐家嘛開始研究母語\footnote{請看\ref{章:相關研究}章},
毋過定定會拄著文字、語料、教材數量無夠的問題。

親像電視台想欲製做母語的新聞,毋過大部份攏是華語的材料;
學生想欲知影一句話,母語按怎講,
%文字空課
%翻譯本身
%隨身翻譯機
%數位資料無夠→翻譯重要
%外地人到臺灣,聽無臺灣話,
這時陣就會當利用華語資料攏誠濟的優勢,
共華語翻譯做母語,按呢就會解決這幾項問題。

這馬翻譯的技術已經發展到一个坎站矣,
毋過華語翻譯做母語的效果攏無蓋好,
主要是母語的語料數量無夠,
本論文就針對華語到閩南語的翻譯,
研究按怎處理數量有限的閩南語語料,予翻譯閣較好。
予後壁的人會使利用這个成果,繼續研究閩南語,
抑是會當利用這篇文章的經驗,推廣到別的臺灣語言,
上直接的,就是會當予閩南語使用者有閣較方便的數位環境。

\section{語料狀況}
\label{節:語料狀況}
佇處理閩南語語料進前,都愛先了解閩南語語料的歷史。

上早的閩南語文體是明國時代流傳落來的荔鏡記戲文\cite{吳守禮《荔鏡記戲文研究──校勘篇》(民國五十年,1961)},
清國時代有閣較濟的字典、歌仔冊佮教會詩歌,
日本時代開始有人感受閩南語消失的威脅,產生出臺灣話文論戰,
到中日戰爭開始,日本人禁止用漢文。%愛閣順過
中華民國政府來臺灣隨就二二八,臺灣文學因為白色恐佈,
到最近二十幾冬,閩南語就開始有大量的文章

因為歷史誠久長,
閩南語的書寫方式有誠濟種,會當分做三種。
第一種全部用漢字,叫做全漢,
就全部用漢字表達閩南語,
因為閩南語有部份毋是漢語,
拄著這種情形逐家用的漢字攏無仝,
到最近幾冬,才有教育部以官方單位規範用字\footnote{臺灣閩南語推薦用字700字表\cite{臺灣閩南語推薦用字700字表},佇96~99年公佈修正}。

第二種是用拼音,
佇清國時期,來傳教的傳教士為著學閩南語,佮鶴佬人講話,
定一套閩南語的羅馬拼音,一般號做「教會羅馬拼音」\footnote{g幾仔个版本,一般是講X的版本}
日本時期,日本政府嘛是為著統治原因,
用日本的aiueo來記錄\footnote{台日、日台大辭典}。
到中華民國時期,
有模仿華語注音符號的方言音符號,
最近幾十年,有主張佮英文系統較倚的通用拼音,
上尾教育部改教羅羅馬拼音的缺點\footnote{毋是一擺就改好,中央閣有TLPA},
號做「臺灣羅馬字拼音」,後壁號做「臺羅」。
因為這馬羅馬拼音較有優勢,所以本論文共全部用拼音的文章攏叫做「全羅」。

面頂兩種攏有一个缺點,
就是上深的漢字抑是全篇的拼音對一般人來講較歹接受,
第三種是共漢字佮拼音濫做伙,
主要是漢字,拄著較歹寫抑是揣無本字的漢字,
就寫拼音。

\section{論文架構}
\label{節:論文架構}

這篇論文的架構是。第\ref{章:相關研究}章介紹佮臺灣語言有相關研究,閣有翻譯原理
第\ref{章:翻譯語料}章介紹語料型式的影響,
第\ref{章:摻猶未整理語料}章介紹無仝語料的性質,而且按怎利用因的特性互相整理予語料較完整。
第\ref{章:網路語料庫}章寫按怎分別網路頂的臺語佮華語資料。
上尾的結論佮以後會當發展的方向攏記佇第\ref{章:結論佮未來發展}章。

\chapter{相關研究}
\label{章:相關研究}
拼音系統
	臺羅
	方言音
閩南語系聲韻學
語音學
語音
變調
翻譯

語料庫


\chapter{翻譯語料}
\label{章:翻譯語料}

頂一章有講

\section{新聞語料庫}
\label{節:新聞語料庫}
華語臺語雙語語料庫\footnote{\url{http://icorpus.iis.sinica.edu.tw/}}(後壁用「新聞語料庫」稱呼)是何澤政\footnote{一九七零年代出世,臺中烏日人}對民國九十七年十一月初六開始,逐工揣兩篇華語新聞,先斷句,後尾翻譯做臺語教會全羅。親像原本的新聞「這幾天寒流再度發威」,翻譯做「tsit4-kui2-kang han5-liu5 koh-tsai3 tian2-ui」(這幾工寒流閣再展威)\footnote{原文是教會羅馬字,為著文章一致,以教育部的臺羅書寫}。
新聞語料庫的翻譯罕得改變用詞的先後,親像面頂的「這幾天寒流再度發威」,較袂翻做「寒流這幾工閣再展威」,除非照華語用詞先後翻譯的結果無順,才會調整。

澤政佇語料內底用
「挕捒 hinn3-sak4」\footnote{挕捒意思共物件擲掉、放捒,嘛就是華話的「丟棄」,例句甲(家己做的):這支筆好好,為啥物愛共伊挕捒。例句乙(Tek-hôa,Nah ē teh批判民進黨?,\url{http://taioan-chouhap.myweb.hinet.net/089.htm}):民進黨接續李登輝的路線, 繼續加強黨國時代權力者, 倚附者佮既得利益者的優勢,毋是共刜挕捒。}
、
「作孽」\footnote{
青-少-年|tshing1-siau3-lian5 作-孽|tsok4-giat8 跤-踏-車|kha1-tah8-tshia1 擲|tan3 落|loh8 河-中|ho5-tiong1。

愛耍手賤,華語的「惡作劇」。例句甲:叫你莫摸你閣摸,誠實手賤愛作孽!(家己做的)例句乙:你這个作孽囡仔,你是按怎沐甲一身軀烏趖趖?(惠光,天真瀾漫,\url{http://ip194097.ntcu.edu.tw/nmtl/DADWT/thak.asp?id=992})}
本土的詞以外,伊嘛會配合這馬發生的代誌,用較時行的臺語,親像
「喙罨」\footnote{華語的「口罩」}、
「心肌梗窒」\footnote{心|sim1 肌|ki1 梗|king2 窒|that4,華語的「心肌梗塞」}、
「自來水」\footnote{臺語較古典的用法,會號做「水道水」}、…。
而且除了現代臺語,澤政伊閣會去查台華線頂辭典\footnote{\url{http://ip194097.ntcu.edu.tw/iug/Ungian/SoannTeng/chil/Taihoa.asp}}
選擇較古典的用詞\footnote{台華線頂辭典是古早語料,一个詞若台華線頂辭典查有,教育部辭典查無,就當做伊是較古典的詞},親像
「𤺪|sian7 篤-篤|tauh4-tauh4」\footnote{熱-天|juah8-thinn1 高-溫|ko1-un1 炎-熱|iam7-juah8 規-工|kui1-kang1 𤺪|sian7 篤-篤|tauh4-tauh4 。|。}、「鬥-贊-手|tau3-tsan3-tshiu2」\footnote{佇|ti7 三|sam1 一-一|it4-it4 大-地-動|tua7-te7-tang7 慷-慨|khong2-khai3 鬥-贊-手|tau3-tsan3-tshiu2 ,|,}。

拄著外來詞,澤政嘛會選擇保留原文,拄著華語的「歐巴馬」佮「西藏」,會翻轉去英文「Obama」、「Tibet」。

斷詞組

加漢字,

若源頭是日本話外來詞,就會直接用教羅寫出來,請看

\section{翻譯架構}
\label{節:翻譯架構}
華語翻譯到臺語需要兩項物件,一項是華語翻譯到臺語的語詞對照表,另外一項是知影臺語逐句好歹的語言模型。這馬時行的翻譯系統是統計翻譯(statistical machine translation)\footnote{有文字佮樹兩種,這馬講文字的},會當分做三个部份:

第一个是對齊模型(alignment model),負責產生語詞對照表,逐擺提一組華語佮臺語的平行語料出來,親像「我 要 吃飯」和「我 欲 食 飯」,華語詞的「要」,會對應到「我」、「欲」、「食」、「飯」臺語詞,經過大量的平行語料,上尾知影華語的「要」定定對應著臺語的「欲」,也就是共對應頻率懸的組合留落來。

第二部份是語言模型(language model),伊去記錄逐个詞後壁定定會接啥物詞,若有一句話是「…欲 食…」,有「欲」佮「食」兩个詞,咱知影「…欲 食」的後壁接「飯」比「…欲 食」的後壁接「湯」的機率較大,也就是講「欲 食 飯」連紲詞比「欲 食 湯」連紲詞機率大,若語言模型一擺看「欲 食 飯」三个詞,就是三連紲詞模型(3-grams model)。語言模型判斷一句話,伊出現的機率有偌大,就是看這句話伊內底連紲詞的機率是偌大。

上尾一部份是解碼器(decoder),提面頂講的對齊模型、語言模型,來翻譯華語到臺語。因為翻譯的問題毋是多項式時間(NP problem)會當解出來的,所以解碼器袂使硬算全部的可能,必須用有效率的演算法來翻譯。

\section{評分方式}
\label{節:評分方式}

翻譯大部份攏用BLEU(Bilingual Evaluation Understudy)來評分\footnote{\url{https://github.com/moses-smt/mosesdecoder/blob/master/scripts/generic/multi-bleu.perl}},伊用連紲詞的概念來評分,$BLEU=100\times{e^{\max{0,\frac{\textit{結果-答案長度}}{\textit{結果長度}}}}}\times{\sum_{n=1}^{4}(\textrm{n連紲詞})^{\frac{1}{4}}}$。

準若翻譯的答案是「這 幾 工 寒流 閣再 展威」,咱有兩个翻譯的結果,翻譯結果一「這 幾 工 寒流 有 展威」佮結果二「寒流 這 幾 工 閣再 展威」,請看表\ref{表:範例BLEU分數},答案有「這 幾 工」、「幾 工 寒流」、「工 寒流 閣再」佮「寒流 閣再 展威」4个三連紲詞,結果一有出現2个,所以結果一的三連紲詞分數是2/4,結果二有出現1个,分數是1/4。因為結果二無對應的四連紲詞,伊的分數都比結果一低。

\begin{table}
\caption{翻譯結果一「這 幾 工 寒流 有 展威」佮翻譯結果二「寒流 這 幾 工 閣再 展威」對答案「這 幾 工 寒流 閣再 展威」的分數}%
\label{表:範例BLEU分數}
\centering
\begin{tabular}{|c|cccc|c|}
\hline
翻著的數量 & 一連紲詞 & 兩連紲詞 & 三連紲詞 & 四連紲詞 & BLEU分數\\
\hline
結果一 & 5/6 & 3/5 & 2/4 & 1/3 & 53.73\\
\hline
結果二 & 6/6 & 3/5 & 1/4 & 0/4 & 0.00\\
\hline
\end{tabular}
\end{table}


\section{未知詞問題}
\label{節:未知詞問題}
系統結構會當看圖,語言模型用Witten-Bell加discounting的算法,翻譯模型用預設的參數。

訓練語料用新聞語料庫頭前2300篇新聞,攏總57167句,試驗語料用上尾267篇新聞,攏總6954句。按呢無調整語料,直接照伊的斷詞組落去訓練,共結果佮答案一句內底拆做一字一字,用\ref{節:評分方式}的BLEU去算分數,得著70.67分。

詳細看分數歹的原因,是因為傷濟詞組佇訓練語料無出現過,親像提原本試驗語料的華語句「陸續 開放 一百五十項 的 規費」去翻譯,得著「liok8-siok8 khai1-hong3 一百五十項 e5 規費」 ,「一百五十項」無翻譯出來,是因為訓練語料內底無出現過這个詞組,對訓練語料來講,「一百五十項」就是一个未知詞組。但是訓練語料內底有「兩項」佮「一百五十位」的華語詞組,煞無法度提來用。

為著予翻譯的結果閣較好,按算用兩種方式來加強未知詞的處理,頭一个是改變翻譯的單位,共原本斷詞組的語料改做斷詞抑是斷字,寫佇\ref{節:改變語料格式}節。第二个方法仝款照斷詞組翻譯,若拄著未知詞,針對未知詞專工處理,會當看\ref{節:未知詞另外翻譯}節。



\section{改變語料格式}
\label{節:改變語料格式}

頂一節發覺若用詞組當做翻譯的單位,會因為詞組單位傷大,變做真濟詞組無看過。所以咱都共華語佮臺語攏用一字一字做翻譯的單位,共原本的「陸續 開放 一百五十項 的 規費」,變做「陸 續 開 放 一 百 五 十 項 的 規 費」。照按呢共訓練語料變做一字一字,共這句翻譯會當得著「陸|liok8 續|siok8 開|khai1 放|hong3 一|tsit8 百|pah4 五|goo7 十|tsap8 項|hang7 的|e5 規|kui1 費|hui3 ,|,」,效果比原本斷詞組的閣較好,得著82.94分。

語料格式的影響著翻譯的效果,除了斷字,翻譯嘛會使用斷詞做單位,華語用中研院中文斷詞系統(CKIP)\footnote{\url{http://ckipsvr.iis.sinica.edu.tw/}}斷詞,臺語用辭典斷詞\footnote{實際按怎做請看\ref{節:拄好長度斷詞}}。「」提去翻譯會得著「」。斷詞的分數是76.88分,比斷詞組閣較好一寡,毋過小較輸斷字。三个分數整理佇表\ref{表:斷詞組、斷詞、斷字做單位的翻譯分數}。

\begin{table}
\caption{斷字、斷詞佮斷詞組做單位的分數}%
\label{表:斷詞組、斷詞、斷字做單位的翻譯分數}
\centering
\begin{tabular}{c|ccc}
翻譯單位 & 斷字 & 斷詞 & 斷詞組\\
\hline
分數 & 82.94 & 76.88 & 70.67\\
\end{tabular}
\end{table}

\section{未知詞另外翻譯}
\label{節:未知詞另外翻譯}

對頂頭的結果來看,用斷字來做翻譯較袂拄著未知詞的問題。換別的角度來看,準若咱用斷詞組的翻譯模型,拄著未知詞的時陣,這个未知詞會使提予斷字翻譯模型去翻譯,就是講「陸續 開放 一百五十項 的 規費」提予斷詞組模型翻譯,得著「liok8-siok8 khai1-hong3 一百五十項 e5 規費」,閣來共「一百五十項」佮「規費」這兩个詞組切做斷字「一 百 五 十 項」佮「規 費」,閣擲去斷字模型翻譯,流程會當看圖XX,按呢得著84.85分。

毋過按呢閣無夠,\ref{節:改變語料格式}節的結果證明仝一份語料無仝形式會有無仝的結果,所以閣愛看覓佇斷詞、斷字的情形之下,翻譯效果是按怎變化的。

咱做的是華語翻譯到臺語,攏總兩个語言,逐个語言有斷詞組、斷詞佮斷字三種方法,實驗都有$3^{2}=9$組合,分數請看表\ref{表:華語臺語逐種形式,而且未知詞提予斷字模型翻譯的結果}。
全部分數上懸的是華語斷詞組對臺語斷詞組,原因是伊斷的詞組,對訓練翻譯模型需要對齊模型的語詞對照表\footnote{若袂記得,請看\ref{節:翻譯架構}節的說明}有幫助。
而且毋管臺語的狀態,華語斷詞攏比華語斷字閣較好,因為中研院中文斷詞系統會共定定用的詞組當作詞,親像「看 書」因為是定用詞組,會合做伙做「看書」,看無遐爾用著的「看 電視」猶原是「看 電視」。

毋過臺語斷字變斷詞了後,效果煞較禾黑,是因為臺語斷詞干焦用辭典\footnote{實際按怎做請看\ref{節:拄好長度斷詞}}爾爾,而且詳細去比較「華語斷詞-臺語斷字」佮「華語斷詞-臺語斷詞」的結果,6954句內底有1367句無仝,用人工看頭前151組無仝的結果,逐組揀1个較好的。有71組是斷字的結果較好,有56組是斷詞模型較好,賰的24組是口腔無仝,當做是平平仔好。
閣去查斷詞為啥物翻譯較禾黑,詳細看原本斷詞的內容,「遊|iu5 客-人|kheh4-lang5 數|soo3」斷做「遊|iu5 客-人|kheh4-lang5 數|soo3」,都有淡薄仔問題矣,才會拖著翻譯效果。
%會使講準做「華語斷詞-臺語斷字」的分數比「華語斷詞-臺語斷詞」較懸,毋過人來看,煞無一定是按呢。


\begin{table}
\caption{華語臺語逐種形式,而且未知詞提予斷字模型翻譯的結果}%
\label{表:華語臺語逐種形式,而且未知詞提予斷字模型翻譯的結果}
\centering
\begin{tabular}{c|ccc}
\diaghead{\theadfont Diag ColumnmnHead II}%
{華語形式}{臺語形式} & 斷字 & 斷詞 & 斷詞組\\
\hline
斷字 & 82.94 & 82.75 & 80.61\\
斷詞 & 84.27 & 84.05 & 82.89\\
斷詞組 & 84.05 & 83.90 & 84.85\\
\end{tabular}
\end{table}

\section{語料形式選擇}
\label{節:語料形式選擇}
咱若共\ref{節:改變語料格式}節佮\ref{節:未知詞另外翻譯}節的方法合做伙,效果上好的是斷詞組對斷詞組,毋過斷詞組需要用剖析器去揣結構樹,閣來定規則決定詞佮詞啥物時陣愛敆做伙變詞組,這就是另外一門學問矣。目前臺語閣無這資源,自然賰「華語斷詞-臺語斷字」佮「華語斷詞-臺語斷詞」上好,準若有好的斷詞工具,臺語斷詞模型應該愛比臺語斷字模型閣較好,雖然佇這擺實驗斷詞模型顛倒較禾黑,毋過為著未來臺語的斷詞研究方便比較,後壁的實驗模型攏是用「華語斷詞-臺語斷詞」。

\chapter{摻猶未整理語料}
\label{章:摻猶未整理語料}
頂一章使用新聞語料庫,語言模型嘛是用平行語料的臺語訓練的,其實這馬有一大部份的臺語語料攏毋是平行話料,攏是純臺語一種語言爾爾。
這種純臺語的語料其實嘛會當提來訓練語言模型,毋過語料的形式就有足濟款的,親像有漢羅、全羅佮全漢等等。有的語料會敆兩種以上的文本。
毋過語料的形式無仝,會當利用的部份就無仝。親像全羅會當提供斷詞的資訊,有漢字佮拼音一對一的當提來做辭典。
這章希望會加入新的語料庫,而且利用⿰因無仝款的性質,來互相整理,翻譯的效果閣較好。


\section{教育部辭典佮數位典藏}
\label{節:教育部語料佮數位典藏}
\subsection{教育部辭典}
\label{節:教育部辭典}
這章按算加入教育部辭典佮數位典藏兩个語料庫,教育部辭典全名「臺灣閩南語常用詞辭典」\footnote{\url{http://twblg.dict.edu.tw/holodict_new/}},正式版是100年上線,伊有25892的詞條\footnote{1021230申請到的版本},內底誠濟生活的用語,大部份詞條攏有漢字、音標、解釋、佮例句。的翻譯。
嘛因為伊有漢字佮音標,就會當共遮的漢字佮音標收集起來,當做用字參考的字典,
這个辭典是教育部編的,當然漢字有照教育部家己的規範\footnote{臺灣閩南語推薦用字700字表,佇96~99年公佈修正}來寫,所以伊內部的用字前後有一致,為著翻譯的效果佮使用者的方便,就共教育部辭典的用字當做標準,若有用字佮教育部的用字無仝的,就改做教育部的用字。//閣愛順過

伊的例句,除了臺語漢字佮音標以外,閣有敆華語的翻譯,親像表\ref{表:教育部辭典例句},漢字佮音標的對應會使提去做用字參考的字典,音標的部份會當提來斷詞,訓練語言模型,臺語佮華語的對應會使提來做平行語料。這例句的語料是非常完整,攏總有8XXX句。
\begin{table}
\caption{教育部辭典例句}
\label{表:教育部辭典例句}
\centering
\begin{tabular}{l}
彼个查某囡仔真媠。 \\
Hit ê tsa-bóo gín-á tsin suí.\\
那個女孩子很漂亮。\\
\end{tabular}
\end{table}

除了一般的詞條以外,教育部辭典嘛有收一寡俗語、臆謎猜,攏做388句做附錄句。
因為附錄句干焦提供解釋,所以無法度提來做平行語料,毋過會使提來訓練語言模型。

台語文數位典藏資料庫\footnote{\url{http://xdcm.nmtl.gov.tw/dadwt/pbk.asp}}(下跤號做數位典藏)是國家臺灣文學館收集1885~2006年的語料,攏總2167篇。
照時代分做清國時期170篇、日治時代490篇,民國統治1507篇。內底嘛有照語料形式分做四類,有詩387條、散文1127篇、小說387篇、劇本49篇。攏總416343句。

因為伊是百外冬的語料庫,較古早的語料,用詞就較古典,//閣再加

\subsection{數位典藏}
\label{節:數位典藏}
數位典藏的來源語料百百款,臺文館⿰因為著格式統一,就替原底是全漢抑是漢羅的語料補全羅拼音\footnote{有時陣劇本邊仔的解釋會用漢字,親像「(福哥仔出場)」},若原本是全羅,就請人拍漢羅,會當看表\ref{表:數位典藏語料},拍漢羅的時,⿰因若知影漢字,就會拍漢字,賰的外來語,抑是較本土的語詞就會用音標來拍。
伊有的語料是一句對齊一句,有的是一段對齊一段。

\begin{table}
\caption{數位典藏語料漢羅、全羅對照}
\label{表:數位典藏語料}
\centering
\begin{tabular}{l}
漢羅:Koh m7知u7危險.........., \\
全羅:Koh m7-tsai u7 gui5-hiam2..............., \footnote{數位典藏內底原本是白話字(教會羅馬拼會),為著讀者方便,攏改做教育部的臺羅}\\
\end{tabular}
\end{table}

\section{拄好長度斷詞}
\label{節:拄好長度斷詞}
新聞語料庫是斷詞組,為著翻譯的效果需要改做斷詞,啊拄好教育部辭典佮數位典藏有全羅標記斷詞的狀況,就會使提來幫助新聞語料庫斷詞。

斷詞的標準有誠濟種,為著方便,以教育部的「臺灣閩南語羅馬字拼音方案連字符使用原則」的連字符當做一个詞,若「tsiah8 png7」(食白米飯的意思),當做兩个詞,「tsiah8-png7」(食物件的意思),當做一个詞。

定看著的斷詞方法有上長詞優先\footnote{(FMM)},伊的做法是自頭開始,看頭前幾个字是毋是會當揣著一个佇辭典的詞,若會使,就揀上長的彼个,…%解釋比如說,『我想要吃飯』可以切成『我,想,要,吃,飯』『我,想要,吃飯』『我,想要吃飯』『我想要,吃飯』『我想要吃飯』,其中,能夠在字典找到詞的切割方式有『我,想,要,吃,飯』『我,想要,吃飯』,
%ㄍㄛˊ
因為上長詞有時陣會揀著無好的組合,親像「」「」
%j揣一个1+3比2+2較歹的例
為著閃避這種情形,莫予長詞搶短詞的字,咱就用「拄好長度斷詞」。斷詞的方法是佮上長詞優先相像,詞仝款愈長愈好,毋過咱予無仝長度的詞無仝分數,設定上長四字詞,一字詞1分、兩字詞1/2分、三字詞1/3分、四字詞1/4分,閣用維特比(Viterbi)揣分數上低的斷詞切法。親像「頭前 有 一張 椅仔」就是1/2+1+1/2+1/2=3分,若「」%用面頂2+2的例

毋過輸入的資料若是全羅「hoo7 i1 tsut4-khi3 sng2」,上好的答案是「hoo7 i1 tsut4-khi3 sng2/予伊出去耍」,但是拄好長度的斷詞煞會斷出「hoo7-i1 tsut4-khi3 sng2/雨衣出去耍」,這个情形上長詞優先嘛會拄著。毋過若有提供漢字,就袂拄著這種情形。

新聞語料庫的華語部份用中研院中文斷詞系統(CKIP)\footnote{\url{http://ckipsvr.iis.sinica.edu.tw/}}斷詞

\section{漢羅全羅對齊}
\label{節:漢羅全羅對齊}
佇\ref{節:教育部語料佮數位典藏}節有講著,數位典藏是提供漢羅佮全羅的對照,因為咱的翻譯需要一个漢字對一个音標的一對一,所以愛共數位典藏伊原本一段對齊一段的語料改做一字對一字。
而且愛注意數位典藏佇2006年完成,教育部的漢字規範對2007年才公佈,所以⿰因兩个的用字規範嘛是無仝款的。
毋過數位典藏的語料倩人整理的時陣內部有訂標準,伊的漢字有一半以上攏是會用得的,而且本論文以教育部的為主,所以
對齊的做法是共全羅逐字攏去對看覓漢羅,看佗一个組合會佇字典內底



%有1個少年人;伊抵tng7 teh 想
%U7 chit8 e5 siau3-lian5 lang5; i tu2-tng7 teh siuN7 phok-su7 lun7-bun5, 

\section{漢羅全羅轉一對一}
\label{節:漢羅全羅轉一對一}
共漢羅全羅一字一字對齊了後,會發覺一个問題,有的字是一對一,有的字煞干焦音標爾。
翻譯格式一致會予效果閣較好,所以就愛共干焦音標的漢字補起來變一對一。頭一步就是用\ref{節:拄好長度斷詞}節的方法來斷詞,因為有的詞可能一字一對一、一字音標,親像「彰化」,寫做「tsiong1化」,按呢佇辭典內底加逐種可能,閣愛加「彰hua3」、「彰tsiong1 hua3」、…攏總九種\footnote{漢字、音標、一對一三種兩字,攏總$3^{2}=9$種}。為著查字典的速度閣較緊,就親像圖XX仝款,逐个詞一字一字處理落來,逐字分做漢字、音標、一對一三个點,第二个字閣佇這三个點閣生落去,毋過第一个字有佮別的詞仝款,就會使公家一个點,親像「彰化」「將來」「將軍庄」,因為限制上長四字詞\footnote{照教育部的「臺灣閩南語羅馬字拼音方案連字符使用原則」,有可能有五字詞,毋過這擺實驗限制四字詞},一个詞上濟產生120點\footnote{第一層加到第四層,$3^{1}+3^{2}+3^{3}+3^{4}=120$},毋過揣候選詞的時間複雜度是$O(1)$。

決定斷詞斷佇佗位了後,逐个斷詞的所在可能有超過一个的候選詞,「彰化的米誠好食」,
上尾閣用語言模型,配合維特比算法,揀出機率上懸的詞組。

\section{整理實驗流程佮結果}
\label{節:整理實驗流程佮結果}

因為新聞語料庫有一對一需要斷詞,數位典藏有斷詞毋過愛轉一對一\footnote{因為有一部份是干焦漢字,所以嘛愛標音標},教育部辭典斷詞佮一對一攏有。
攏就會使用教育部辭典佮數位典藏共新聞語料庫斷詞,閣共斷好的新聞語料庫佮教育部辭典提來標數位典藏的一對一,閣重做幾仔擺,到收斂為止,親像圖XX。

實驗語料除了\ref{節:未知詞問題}節分配新聞語料庫,嘛加入教育部辭典的例句\footnote{原本8XXX句,替換腔口詞了變34693句,詳細請看附錄\ref{章:腔口統計佮處理}}。
所以訓練語料是用新聞語料庫頭前57167句佮教育部31200句,
試驗語料是新聞語料庫後壁6954句佮教育部3493句例句。
語言模型除了用訓練語料以外,閣有加教育部附錄句388句。

拍分數以詞為單位

\begin{table}
\caption{新聞語料庫佮數位典藏互相整理的實驗結果}
\label{表:互相整理實驗結果}
\centering
\begin{tabular}{lcccc}
整理幾擺 & 0\tablefootnote{新聞只用教育部辭典斷詞} & 1 & 2 & 3\\
語言模型無加典藏 & 50.57 & 53.91 & 53.91 & 53.91\\
語言模型加典藏 & 49.50 & 51.94 & 51.97 & 51.97\\
\end{tabular}
\end{table}
 佇使用仝款的語料時

典藏佮新聞有互相整理過,分數有較懸
 摻典藏做語言模型


分數降落來
因為典藏對訓練佮試驗語料來講是外部的資料
 分數一息仔就收斂



新聞語料做第一擺就收斂
典藏到第二擺就收

\chapter{網路語料庫}
\label{章:網路語料庫}
加入新聞語料庫、教育部辭典佮數位典藏了後,按呢華臺平行語料有98814句\footnote{新聞語料庫64121句,教育部辭典34693句},會當訓練語言模型的臺語有\footnote{新聞語料庫64121句,教育部辭典例句34693句、附錄句388句,數位典藏416343句},這个數量對照別種語言語料庫的數量也是小可嫌少。
佇這个網路的時代,收集語料上緊的方法就是去網路面頂掠。看圖XX,先共臺語專門的字詞擲去搜尋引擊\footnote{親像Google、Bing},閣照揣著的網頁去掠相關的臺語。
臺語的網頁內底除了臺語以外,有可能閣濫一部份的華語,為著莫予華語語料影響著臺語模型,所以愛想辦法共臺華兩種語言分開。分開了後

%圖:關鍵詞 引擊 網址 掠網頁 網頁html 轉文字 一句一句的語料 判斷語言 臺語/華語 對齊

\section{TGB通訊-臺灣組合}
\label{節:TGB通訊-臺灣組合}
TGB通訊\footnote{\url{http://taioan-chouhap.myweb.hinet.net/}}是學生台灣語文促進會對1999年10月開始\footnote{\url{http://taioanchouhap.pixnet.net/blog/post/32374696}}一個月一期的刊物。頭前60期以臺語為主,61期有提供華語對照,有時陣臺語句內底會濫一寡華語詞,形式較無固定

因為有臺語、有華語佮臺華平行語料,嘛閣有濫做伙的, 誠濟種實際會拄著的情形會當提來做判斷語言的語料。
有的時陣臺語佮華語濫做伙,先定義啥物情形算臺語,啥物情形算華語

用人工分

主要語句是臺語,有一兩个詞是華語嘛是算臺語
臺語

聽人講 khah 早有出現過『小蜜蜂』
我 beh tńg 來種作 ! ── 記 0312 Truku 反亞泥 ‧ 還我土地運動
有台灣味 ê 繪本──《我和我的腳踏車》 .

華語
華語臺語攏通的算華語
有華語完整的一句話就算華語
毋是華語嘛毋是臺語,親像英文、日語、…

「 糟了 ,是工地火燒厝, 緊轉去打 火 ! 」建設公司 的 工地主任 從手機接到消息,通話結束後就帶著那群混混先離開了。
『聽說妳最近遇到什麼問題 , 是不是 ? 怎麼了 ? 』好性地 ê QA 繼續問--落-去 .
去越南胡志明市 4 工/越南胡志明市四日行 @Gio̍k-hōng


\section{語言判斷特徵}
\label{節:語言判斷特徵}
為著予電腦會當分別臺語佮華語,咱就愛準備幾項臺語佮華語無仝的特徵。
臺語佮華語上大差別就是用詞無仝,臺語寫「食飯」、「無法度」,華語寫「吃飯」、「沒辦法」,
所以咱揀定用詞出來,當作咱的特徵之一。
毋過臺語佮華語有誠濟共同詞,親像「火車」、「電腦」,⿰因寫法是仝款的,所以咱袂使直接提定用詞來做,因為內底會有共同詞,所以咱愛對臺語定用詞內底揀華語袂用的特徵詞出來,華語嘛仝款愛揀出臺語袂用的特徵詞。

選特徵詞的方法是先統計\ref{節:整理實驗流程佮結果}的試驗語料佮數位典藏,揀出頭前15000个上定出現的臺語定用詞\footnote{有標點符號}
華語部份嘛仝款,佇中央研究院現代漢語標記語料庫\footnote{\url{http://app.sinica.edu.tw/cgi-bin/kiwi/mkiwi/kiwi.sh}}內底揣15000个上定出現的華語定用詞,
閣來對上定用的臺語定用詞開始,若這个定用詞的漢字詞無出現佇華語15000定用詞內底,就共伊當做特徵詞。
若伊出現佇華語定用詞內底,就莫治。
就按呢揀出頭前7000个特徵詞。
華語的部份嘛仝款,揀出7000个無佇臺語定用詞的特徵詞。

%看圖

有臺語佮華語的特徵詞了後,咱對網頁整理出一段一段的語料,先用\ref{節:拄好長度斷詞}節的臺語斷詞,看斷詞出來的結果,佇臺語7000个特徵出中,分別出現幾个。
除了7000个特徵詞,咱閣用臺語語言模型分數,斷詞了的全部詞數,1~4字詞分別數量\footnote{「我 想 欲 食飯」就有3个一字詞,1个兩字詞,全部4个詞},按呢干焦臺語就有7006个特徵,摻華語就有14012个特徵。

\section{分類器}
\label{節:分類器}

\subsection{PCA}
\label{節:PCA}

\subsection{LDA}
\label{節:LDA}

\subsection{支持向量機}
\label{節:支持向量機}

\subsection{DNN}
\label{節:DNN}


\section{判斷語言實驗結果}
\label{節:判斷語言實驗結果}
這節實驗的語料是對TGB通訊創刊開始,到2014年6月12日為止攏總177期1179篇文章,
提出頭前1000篇做訓練語料,臺語有9368段488844詞,華語有8519段439436詞;
後壁179篇做訓練語料,臺語有1344段75282詞,華語有2397段114901詞。
以段做辨識單位,提來做SVM,錯誤率是 $\%$ 。

毋過14012个特徵實在是傷濟矣,所以咱試看覓共7000特徵詞減少,看會影響著辨識率無。
而且PCA佮LDA是轉特徵空間到較細的空間,來試配合SVM效果會按怎,結果會當看圖\ref{圖:無仝分類模型佮特徵詞數量對分類臺華語效果的影響}。
對結果來看SVM比LDA效果閣較好,用PCA對SVM效果無啥物傷大的影響。
而且對50~100个特徵詞了後,加閣較濟的特徵詞,攏袂影響的辨識的效果。

\begin{figure}
\begin{tikzpicture}
\begin{axis}[
scaled y ticks=real:1,
ytick scale label code/.code={},
ymax = 800,
symbolic x coords={0,10,20,50,100,200,500,1000,2000,5000,7000},
xtick=data,
height=12cm,
width=16cm,
grid=major,
xlabel={特徵詞數量},
ylabel={分類錯誤數量},
legend style={
cells={anchor=east},
legend pos=north west,
%mark size=0.5em
}
]
%     	SVM	LDA	LDA加SVM	PCA加SVM
%0	494	377	351	494
%10	251	379	352	251
%20	191	386	351	191
%50	147	380	217	148
%100	145	382	217	149
%200	152	376	215	150
%500	159	468	206	150
%1000	156	506	253	155
%2000	155	507	301	165
%5000	156	673	445	165
%7000	158	742	543	169
\addplot+[mark=*,mark size=0.5em] coordinates {
(0,494) (10,251) (20,191) (50,147) (100,145) (200,152) (500,159) (1000,156) (2000,155) (5000,156) (7000,158) 
};
\addplot+[mark size=0.5em] coordinates {
(0,377) (10,379) (20,386) (50,380) (100,382) (200,376) (500,468) (1000,506) (2000,507) (5000,673) (7000,742) 
};
\addplot+[mark size=0.5em] coordinates {
(0,351) (10,352) (20,351) (50,217) (100,217) (200,215) (500,206) (1000,253) (2000,301) (5000,445) (7000,543) 
};
\addplot+[mark size=0.5em,mark=triangle*,red] coordinates {
(0,494) (10,251) (20,191) (50,148) (100,149) (200,150) (500,150) (1000,155) (2000,165) (5000,165) (7000,169) 
};

\legend{SVM,LDA,LDA加SVM,PCA加SVM}
\end{axis}
\end{tikzpicture}
\caption{無仝分類模型佮特徵詞數量分類3741段臺華語的效果}
\label{圖:無仝分類模型佮特徵詞數量對分類臺華語效果的影響}
\end{figure}


\chapter{結論佮未來發展}
\label{章:結論佮未來發展}
目前臺灣的本土語言佇沓沓仔流失,需要逐家用心來注意,這嘛是這篇論文研究華語翻譯到臺語的上主要目的。
為著逐家後壁研究的方便,本論文研究的過程佮結果,全部公開佇網路頂\footnote{\url{https://github.com/sih4sing5hong5/huan1-ik8_gian5-kiu3.git}},詳細按怎用請看附錄\ref{章:按怎裝程式}。
除了臺語以外,客話佮咱的原住民南島語,嘛會使利用這个研究成果。
若是閣配合語音模型\footnote{請看附錄\ref{章:語音模型}},都會使做一个即時口語翻譯系統。
若第一線佇學校教冊的老師需要電子化的翻譯,抑是關心母語的文字工作者,攏會使利用這个研究成果繼續做落去。

後壁的XX章節是針對臺語所寫的研究方向。XX是客話、南島話攏會使參考的…
\section{校對資料}
\label{節:校對資料}
面頂的實驗結果共咱講就算咱提著的語料毋是蓋完整,咱嘛會當利用改伊的款,抑是用有一對一較完整的資料去鬥處理,攏會使予翻譯的效果閣較好。
毋過完整的資料數量若較濟,翻譯的效果會愈好。

準做資料是親像數位典藏仝款補一部份漢字爾仔,嘛是愛人工閣巡一擺,其他漢羅抑是全羅的用電腦自動標一對一了後,閣較需要人工共⿰因閣校對過。
愛有完整一對一資料,人工是走袂去的,親像圖X仝款,咱有完整的資料,去標猶未整理的語料,經過人工檢查了後,完整的資料就閣較濟,按呢標一對一就閣較準,就無需要遮爾濟的人工,創造一个好的循環。

毋過一開始人工檢查是開錢開時間開氣力的代誌,若準做有工具,予標一對一的正確率閣較懸,就會減少人工檢查的負擔。
人工檢查前的輸入佮人工檢查了的輸出,嘛會使做一个改錯字的系統,予人免一直改仝款的錯誤。

毋管是標一對一,抑是下跤的斷詞、剖析佮語音資料,發展技術佮照顧語料對發展臺語研究來講平平仔重要,絕對袂使重視技術煞袂記得語料。
%\subsection{重斷數位典藏}
%\label{節:重斷數位典藏}
%數位典藏有的斷詞其實毋好,親像「」應該是兩个詞

\section{斷詞}
\label{節:斷詞}
%看楊允言
華語斷詞是一个發展足完整的技術\footnote{CKIP正確率%},毋過臺語的語料毋華語遐爾濟的人工,就會影響著效果。
本論文是用對「上長度優先」的算法改做「拄好長度斷詞」\footnote{請看\ref{節:拄好長度斷詞}節},嘛會使閣用統計方法看斷詞的效果會較好無。
用統計的方法會使用翻譯工具來做,一字一字斷字的輸入配合一詞一詞斷詞的輸出,提去訓練翻譯模型,按呢就有一个統計的斷詞工具。
到底佗一个方法佇臺語、客話這種幾萬、幾十萬句語料,效果會閣較好,就需要閣一寡研究矣。
\section{剖析}
\label{節:剖析}
%看楊允言
楊允言教授捌做過臺語的剖析,毋過需要語言學的知識,閣有一致的人工檢查,而且一開始的語料歹收集,可能著的先共臺語翻譯做華語,才閣共華語語句的剖析結果\footnote{中研院剖析}對應轉去臺語語句,按呢就有初步的資料矣。
訂好規則了後,就需要訂好剖析的規則
人工若校對,全部的資料著愛收集起來
而且嘛愛親像XX節的XX圖仝款,愛做一个改錯誤的程式,予人工莫一直改仝款的物件
等待資料有夠濟,就會提樹仔的語料來訓練一个臺語的剖析器\footnote{star ford parser}

\section{語音技術}
\label{節:語音技術}
\subsection{語音辨識}
\label{節:語音辨識}
語音辨識就是佮聲音轉做文字,這方面的開源工具的HTK佮Kaldi,HTK發展的時間較早,毋過最近上新的版本是年,

Kaldi是較新的工具,除了使用一部份HTK的程式外,伊閣加入了深層類神經網路\footnote{DNN},

字幕 國語字→臺語字
功德一件

\subsection{語音合成}
\label{節:語音合成}
語音合成就是佮文字轉做聲音,佮語音辨識顛倒反。這方面有HTS,HTS是對HTK修改的合成工具。

%介紹敢愛寫佇遮?
伊需要音檔,標記發音內容的音檔,閣有音類的問題集。設計標仔
標仔的時間會使對HTK訓練
根據經驗,3000句會使,5000句普通,7000句上好

\subsection{臺語變調}
\label{節:臺語變調}
準做想欲共臺語的字變做臺語的聲音,咱必須佮字,先標一對一斷詞,揣出音標。
但是臺語的變調是足複雜的,無法度予HTS內底的決策樹來做
所以愛先用另外一支程式專門來變調
楊允言教授就有做過\footnote{},伊是用規則來確定變調的狀況,毋過伊用的語料無蓋濟,若數量一濟,規則式的變調會較歹處理。
變調嘛會當用機器學習的方法來做,共規則式用掉的特徵做伙下入去,看佗一个分類模型會較好,毋過伊上大的好處就是伊管理方便,免拄著啥物新語料,規則就愛全部重改,伊干焦需要共新語料加入重訓練就好矣,毋過伊的內部試驗就無保證一定著矣。


\chapter{--其他--}
To find $T$, we try to approximate a (real-value) GCD of $D=\left\{
T_{1},T_{2},...,T_{n-1}\right\}  $. This is done by running the DBSCAN
(Density Based Spatial Clustering of Applications with Noise) clustering
algorithm \cite{Ester1996DBSCAN} to group data items and filter out noise,
followed by a testing process to find $T$. There are several parameters used
in this approach:

In the simulations, VISSIM \cite{Mosseri2004VISSIM} was used to simulate
vehicle traffic. The Poisson arrival model with rates 18/20/24 vehicles per
minute was used to generate traffic flows into a signalized road section of
$650$ m from A to B as illustrated in Fig. \ref{fig:f_map} for $40$ minutes,
and the Wiedemann 99 car following model \cite{Mosseri2004VISSIM} was used to
depict the trajectories of vehicles. In the free flow state, vehicles move in
a range of speed between $48$ km/h and $58$ km/h. When encountering a red
signal, vehicles slow down and eventually stop. After the signal turned to
green, vehicles accelerate and then enter the free flow state.
\begin{figure}[pth]
\centerline{\includegraphics[angle=0, width=3.5in,keepaspectratio]
{音/⿳⿳ㆣㄧˋ}} \hfill\caption{The road segment in simulation and field
trial experiments. B is the location of the traffic light.}%
\label{fig:f_map}%
\end{figure}

\bibliographystyle{IEEEtran}
\bibliography{翻譯論文}

\begin{appendices}
\chapter{腔口統計佮處理}
\label{章:腔口統計佮處理}
\chapter{按怎裝程式}
\label{章:按怎裝程式}
看issue
\chapter{收集網路語料}
\label{章:收集網路語料}
\chapter{臺灣言語工具}
\label{章:臺灣言語工具}
\chapter{語音模型}
\label{章:語音模型}
The contents...
\end{appendices}

\end{document}