\chapter{摘要}
臺灣是一個多元文化、多元語言的國家,
講母語、使用母語是人最基本的權利,
不過母語的電腦相關應用卻很少。
臺灣本土語言很多種,
本論文先針對閩南語,
研究閩南語翻譯語料的特性,
除了閩南語本身以外,
也希望研究結果對別的本土語言有幫助。

本論文主要有五個貢獻。
第一個是本論文提出「剛好長度斷詞」的演算法,
雖然有改善「長詞優先」的缺點,
不過效果無明顯,正確率差不到$1\%$。
第二個是提出一個自動整理漢語語料的方法,
讓資訊不完整的語料庫補足資訊,
發揮最大的價值,
BLEU分數從9.30拉到13.82。
第三个貢獻是證明平行語料數量不到十萬句的時候,
增加語料對翻譯的效果影響非常大,
原本64121句加35025句之後,
BLEU分數從13.82提昇到19.33。
第四個貢獻是漢語語料用詞做單位去翻譯的時候,
用斷字翻譯可以解決未知詞問題,
BLEU分數可以從18.64提昇到19.33。
上尾一個貢獻是提出分類兩種漢語的方法,
分類的錯誤率最低達$3.80\%$。

關鍵字:臺灣閩南語、華語、翻譯、語料、斷詞、語言分類
