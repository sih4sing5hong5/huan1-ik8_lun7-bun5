\chapter{摘要}
臺灣是一個多元文化、多元語言的國家。
講母語、使用母語是最基本的權利,
不過母語的電腦相關應用卻很少,
需要加強自然語言處理的研究和語料收集整理。
臺灣本土語言很多種,
本論文是針對閩南語,
研究閩南語翻譯語料的特性,
除了閩南語本身以外,
也希望研究結果對別的本土語言有幫助。

本論文提出一個自動整理漢語語料的方法,
讓資訊不完整的語料庫補足資訊,
發揮最大的價值,
BLEU分數從9.30拉到13.82。
另外證明平行語料數量不到十萬句的時候,
增加語料對翻譯的效果影響非常大,
原本64121句加到99147句之後,
BLEU分數從13.82提昇到19.33。

關鍵字:臺灣閩南語、華語、翻譯、語料、斷詞、語言分類
